% !TEX encoding = UTF-8
% !TEX TS-program = pdflatex
% !TEX root = ../tesi.tex

%**************************************************************
\chapter{Introduzione}
\label{cap:introduzione}
%**************************************************************

Questa tesi descrive l'esperienza e il percorso lavorativo svolto presso l'azienda GRUPPO4 sotto la supervisione di Tobia Conforto.

%**************************************************************
\section{L'azienda}
Gruppo4 è una web agency Padovana. Da oltre vent'anni Gruppo4 ha accumulato competenze e l'esperienza necessaria per fornire soluzioni efficaci e innovative nel settore web. Mediante un modello organizzativo consolidato e certificato sviluppano WebApp che si distinguono per la chiarezza dell'interfaccia utente (UX/UI) e per la loro usabilità.

%**************************************************************
\section{Gli obiettivi del progetto}
L'obiettivo principale di questo progetto consiste nella realizzazione di un componente di interfaccia grafica per il web in Kotlin. La soluzione deve essere una tabella pivot interattiva dove l'utente ha la possibilità di esplorare liberamente i Big Data contenuti al suo interno. Oltre alla realizzazione del componente, questo progetto ha anche lo scopo di valutare l'efficacia di Kotlin nel realizzare interfacce utente per il web in quanto l'azienda è particolarmente interessata alle sue applicazioni nello sviluppo di interfacce utente perchè Kotlin è già utilizzato nello sviluppo di \gls{api} di Gruppo4.

%**************************************************************
\section{Organizzazione del testo}
Riguardo la stesura del testo, relativamente al documento sono state adottate le seguenti convenzioni tipografiche:
\begin{itemize}
	\item gli acronimi, le abbreviazioni e i termini ambigui o di uso non comune menzionati vengono definiti nel glossario, situato alla fine del presente documento;
	\item per la prima occorrenza dei termini riportati nel glossario viene utilizzata la seguente nomenclatura: \emph{parola}\glo;
	\item i termini in lingua straniera o facenti parti del gergo tecnico sono evidenziati con il carattere \emph{corsivo}.
\end{itemize}