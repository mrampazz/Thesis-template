% !TEX encoding = UTF-8
% !TEX TS-program = pdflatex
% !TEX root = ../tesi.tex

%**************************************************************
\chapter{Introduzione}
\label{cap:introduzione}
%**************************************************************

Questa tesi descrive l'esperienza e il percorso lavorativo svolto presso l'azienda GRUPPO4 sotto la supervisione di Tobia Conforto.

%**************************************************************
\section{L'azienda}
Gruppo4 è una web agency Padovana, da oltre vent'anni ha accumulato competenze e l'esperienza necessaria per fornire soluzioni efficaci e innovative nel settore web. Mediante un modello organizzativo consolidato e certificato sviluppano applicazioni web che si distinguono per la chiarezza dell'interfaccia utente (UX/UI) e per la loro usabilità.

%**************************************************************
\section{Gli obiettivi del progetto}
L'obiettivo principale di questo progetto consiste nella realizzazione di un componente di interfaccia grafica per il web in Kotlin. Il componente deve essere una tabella pivot interattiva che permette ad un utente la possibilità di esplorare liberamente i \emph{big data} contenuti al suo interno. Oltre alla realizzazione del componente, questo progetto ha anche lo scopo di valutare l'efficacia di Kotlin nel realizzare interfacce utente per il web in quanto l'azienda utilizza Kotlin nello sviluppo di \gls{api}.

%**************************************************************
\section{Tabella pivot}
L'azienda mi ha fornito una descrizione accurata della tabella pivot, in particolare della sua struttura e delle sue funzionalità. Nelle prossime sottosezioni descriverò gli elementi che la compongono e le possibili interazioni con l'utente.

\subsection{Struttura}
La tabella pivot è stata suddivisa principalmente in due parti ben distinte: le dimensioni e i dati ad esse. Le dimensioni rappresentano le celle di intestazione della tabella. Esse sono presenti su tutti e due gli assi (righe e colonne) e su un singolo asse possono essere presenti molteplici dimensioni. I dati riferiti alle celle di intestazione corrispondono alla seconda parte della tabella pivot e corrispondo a delle semplici celle che contengono valori interi.
% img dimensione

\subsection{Funzionalità}
La tabella deve essere completamente esplorabile da un utente per questo motivo sono presenti due tipi di azioni: sulle dimensioni e sulle singole celle di una dimensione. \\
Ogni cella di una dimensione deve poter essere aperta o chiusa con un semplice pulsante. Mentre ogni cella di una dimensione può essere aperta per renderizzare i figli di quel nodo se essi esistono. 
% img azione dimensione
% img azione cella

%**************************************************************
\section{Organizzazione del testo}
Riguardo la stesura del testo, relativamente al documento sono state adottate le seguenti convenzioni tipografiche:
\begin{itemize}
	\item gli acronimi, le abbreviazioni e i termini ambigui o di uso non comune menzionati vengono definiti nel glossario, situato alla fine del presente documento;
	\item per la prima occorrenza dei termini riportati nel glossario viene utilizzata la seguente nomenclatura: \emph{parola}\glo;
	\item i termini in lingua straniera o facenti parti del gergo tecnico sono evidenziati con il carattere \emph{corsivo}.
\end{itemize}