% !TEX encoding = UTF-8
% !TEX TS-program = pdflatex
% !TEX root = ../tesi.tex

%**************************************************************
\chapter{Introduzione}
\label{cap:introduzione}
%**************************************************************

Questa tesi descrive l'esperienza e il percorso lavorativo svolto presso l'azienda GRUPPO4 sotto la supervisione di Tobia Conforto.

% decidere se mettere queste descrizioni (ci potrebbero stare imo)
%Introduzione al contesto applicativo.\\

%\noindent Esempio di utilizzo di un termine nel glossario \\
%\gls{api}. \\

%\noindent Esempio di citazione in linea \\
%\cite{site:agile-manifesto}. \\

%\noindent Esempio di citazione nel pie' di pagina \\
%citazione\footcite{womak:lean-thinking} \\

%**************************************************************
\section{L'azienda}

GRUPPO4 è una web agency Padovana da oltre vent'anni. In questo periodo GRUPPO4 ha accumulato competenze e l'esperienza necessaria per fornire soluzioni efficaci e innovative nel settore web. Mediante un modello organizzativo consolidato e certificato sviluppano WebApp che si distinguono per la chiarezza dell'interfaccia utente (UX/UI) e per la loro usabilità.

%**************************************************************
\section{L'idea}

GRUPPO4 da oltre dieci anni fornisce una struttura web, creavista, per permettere ai loro clienti di esplorare una grande mole di dati velocemente. Per fare questo GRUPPO4 utilizza una tabella pivot che permette di collegare, mediante relazioni, diverse informazioni. Le tabelle pivot hanno proprio il vantaggio di unire tutte le informazioni all'interno di un database e fornire un'interfaccia facile da utilizzare per esplorare tali dati. \\
Questo tirocinio aveva come scopo quello di sostituire la tabella pivot di creavista, ormai datata, con una nuova componente realizzata con tecnologie innovative e sicure quali.
% spiegare  perchè sono sicure e innovative?

%**************************************************************
\section{Organizzazione del testo}

\begin{description}
    \item[{\hyperref[cap:processi-metodologie]{Il secondo capitolo}}] descrive ...
    
    \item[{\hyperref[cap:progettazione]{Il terzo capitolo}}] approfondisce ...
    
    \item[{\hyperref[cap:descrizione-stage]{Il quarto capitolo}}] approfondisce ...
    
    \item[{\hyperref[cap:analisi-requisiti]{Il quinto capitolo}}] approfondisce ...
    
    \item[{\hyperref[cap:tecnologie-strumenti]{Il sesto capitolo}}] approfondisce ...
    
    \item[{\hyperref[cap:conclusioni]{Nel ottavo capitolo}}] descrive ...
\end{description}

Riguardo la stesura del testo, relativamente al documento sono state adottate le seguenti convenzioni tipografiche:
\begin{itemize}
	\item gli acronimi, le abbreviazioni e i termini ambigui o di uso non comune menzionati vengono definiti nel glossario, situato alla fine del presente documento;
	\item per la prima occorrenza dei termini riportati nel glossario viene utilizzata la seguente nomenclatura: \emph{parola}\glsfirstoccur;
	\item i termini in lingua straniera o facenti parti del gergo tecnico sono evidenziati con il carattere \emph{corsivo}.
\end{itemize}