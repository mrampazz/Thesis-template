% !TEX encoding = UTF-8
% !TEX TS-program = pdflatex
% !TEX root = ../tesi.tex

%**************************************************************
\chapter{Processi e metodologie}
\label{cap:processi-metodologie}
%**************************************************************

In questo capitolo verrà fornito una descrizione dei metodi e dei processi messi in atto durante il tirocinio, in particolare riguardo i seguenti argomenti: metodologia agile, programmazione funzionale e il concetto di Dataflow dell'applicazione.

%**************************************************************
\section{Metodologia Agile}
Per lo sviluppo del prodotto è stato deciso di utilizzare una metodologia agile in modo da reagire velocemente a problemi e a cambiamenti così da migliorare ed l'efficienza nella realizzazione della componente. L'azienda ha deciso di utilizzare una metodologia agile simile a SCRUM. Infatti applicare nella sua interezza il metodo SCRUM sarebbe stato impossibile dato il ristretto numero di sviluppatori nel team di sviluppo. \\
Le caratteristiche principali della metodologia agile applicata per la realizzazione di questo progetto sono le seguenti:
\begin{itemize}
	\item \textbf{Modello incrementale}: vengono realizzati rilasci multipli e successivi che aiutano a definire più chiaramente i requisiti più importanti dato che essi verranno implementati per primi. Ogni rilascio corrisponde ad una parte funzionante di applicazione;
	
	\item \textbf{Modello iterativo}: un modello iterativo ha la caratteristica di avere una maggior capacità di adattamento in seguito a problemi di implementazione e cambiamenti nei requisiti da parte del cliente;
	
	\item \textbf{Organizzazione in sprint di sviluppo}: il processo[controlla se è il termine giusto] di codifica viene suddiviso in sprint di sviluppo, data la breve durata del tirocinio curriculare essi avranno una durata di circa 4-5 giorni;
	
	\item \textbf{Backlog}: 
		\begin{itemize}
			\item \textbf{Product Backlog}: rappresenta i requisiti e le funzionalità del prodotto definiti mediante le User Stories;
			\item \textbf{Sprint Backlog}: rappresenta l'insieme delle User Stories da realizzare nello sprint indicato;
		\end{itemize}
	
	\item \textbf{User Stories}: l'idea alla base di uno sviluppo agile è la realizzazione delle User Stories. Ogni user story è definita da una \textbf{descrizione del problema} e da una \textbf{priorità}.
	
	\item \textbf{Riunioni}:
		\begin{itemize}
			\item \textbf{Sprint planning}: per pianificare il lavoro da svolgere durante lo sprint;
			\item \textbf{Sprint review}: riunione retrospettiva per verificare il lavoro svolto durante lo sprint;
			\item \textbf{Backlog refinement}: per aggiungere nuove User Stories o migliorare e/o modificare User Stories già create;
			\item \textbf{Riunioni giornaliere}: per verificare lo svolgimento del lavoro, in questo tirocinio sono state sostituite con comunicazioni telematiche giornaliere.
		\end{itemize}
\end{itemize}
% scrivere descrizione più accurata della metodologie agili applicate

\section{Programmazione funzionale}
La programmazione funzionale è un paradigma di programmazione dichiarativa dove un programma è costituito dall'applicazione e dalla composizione di funzioni. In questo progetto si è utilizzata, dove possibile, la programmazione funzionale in particolare mediante le \emph{Funzioni di ordine superiore} fornite da Kotlin. Queste funzioni sono molto utili perchè permettono di scrivere codice più leggibile, conciso e soprattutto hanno la caratteristica di evitare \emph{side-effects}. Come definito dalla documentazione di Kotlin (bib: https://kotlinlang.org/docs/reference/lambdas.html), le funzioni scritte nel linguaggio Kotlin sono considerate come \emph{first-class} quindi esse possono essere contenute in variabili e strutture dati, possono essere passate come argomento di altre funzioni e possono essere ritornate da altre \emph{Funzioni di ordine superiore}. Queste funzioni sono state usate estensivamente nella realizzazione del "parser" descritto nel capitolo 4:descrizione-stage.

\section{Versionamento della soluzione}
\subsection{Git}
Git è un VCS (Version Control System) distribuito che permette di tenere traccia delle modifiche in un prodotto software e di organizzarne la codifica. In questo tirocinio è stata usata la tecnica del feature-branch: ogni aggiunta di feature corrisponde all'apertura di un nuovo branch che deve essere poi approvato previa verifica prima di essere unito al branch \verb|master|.

\subsection{Gitlab}
Gitlab è uno strumento web che permette di implementare un DevOps lifecycle che fornisce una gestione di repository git, un ITS (Issue Tracking System) e altri strumenti quali la "Continuous integration" e "Continuous deployement". Per lo sviluppo di questo progetto mi è stato fornito l'accesso al server privato aziendale di Gitlab.

\section{Ambiente di sviluppo locale}
\subsection{IntelliJ Idea}
IntelliJ IDEA Community Edition è una IDE realizzata da JetBrains che fornisce funzionalità di supporto per lo sviluppo di molti linguaggi, specialmente Kotlin. Questa IDE è stata vivamente consigliata dal mio tutor aziendale per lo sviluppo in Kotlin rispetto ad altri editor per molti vantaggi come l'autocompletamento e la possibilità di eseguire un refactoring automatico di funzioni e classi.
 
\subsection{Gradle}
Gradle è uno strumento di \emph{build automation} per molti linguaggi tra cui Kotlin e Java. Gradle è stato usato per la gestione e l'installazione delle dipendenze del componente, il file di configurazione di Gradle è \verb|build.gradle.kts|, al suo interno sono definite le seguenti dipendenze:
\begin{itemize}
	\item \verb|stdlib-js|:;
	\item \verb|kotlin-react|:;
	\item \verb|kotlin-react-dom|:;
	\item \verb|react|:;
	\item \verb|react-dom|:;
	\item \verb|kotlin-redux|:;
	\item \verb|kotlin-react-redux|:;
	\item \verb|kotlin-extensions|:;
	\item \verb|kotlinx-serialization-core|:;
	\item \verb|kotlinx-coroutines-core|:.
\end{itemize}

\subsection{Organizzazione del lavoro}
Per l'organizzazione del lavoro, in particolare per la gestione dei macro obiettivi di ogni sprint, ho utilizzato l'ITS fornito da Gitlab che fornisce un'interfaccia Kanban che permette di categorizzare in modo efficace le singole attività da realizzare.























