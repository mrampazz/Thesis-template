% !TEX encoding = UTF-8
% !TEX TS-program = pdflatex
% !TEX root = ../tesi.tex

%**************************************************************
\chapter{Processi e metodologie}
\label{cap:processi-metodologie}
%**************************************************************

In questo capitolo verrà fornito una descrizione dei metodi e dei processi messi in atto durante il tirocinio.

%**************************************************************
\section{Processo sviluppo prodotto}
Per lo svolgimento del prodotto da realizzare è stato deciso di utilizzare una metodologia agile in modo da reagire velocemente a problemi e/o imprevisti per migliorare ed ottimizzare l'efficienza nella realizzazione della componente. L'azienda ha deciso di utilizzare una metodologia agile simile a SCRUM.
\subsection{Metodologia Agile}
Le caratteristiche principali della metodologia agile applicata per la realizzazione di questo progetto sono le seguenti:
\begin{itemize}
	\item \textbf{Modello incrementale}: vengono realizzati rilasci multipli e successivi che definiscono più chiaramente i requisiti più importanti e che rappresentano parti funzionanti di applicazione che aiutano nello sviluppo di una struttura armonica e completa;
	
	\item \textbf{Modello iterativo}: un modello iterativo ha la caratteristica di avere maggior capacità di adattamento in seguito a imprevisti e/o cambiamenti nei requisiti da parte del cliente.
	
	\item \textbf{Organizzazione in sprint di sviluppo}: il processo[controlla se è il termine giusto] di codifica viene suddiviso in sprint di sviluppo dalla durata di circa 4-5 giorni per permettere la realizzazione di una riunione di Sprint Planning e una di Backlog Refinement;
	
	\item \textbf{Backlog}: 
		\begin{itemize}
			\item \textbf{Product Backlog}: rappresenta i requisiti e le funzionalità del prodotto;
			\item \textbf{Sprint Backlog}: rappresenta l'insieme delle user stories da realizzare nel prossimo sprint;
		\end{itemize}
	
	\item \textbf{User Stories}: l'idea di base di uno sviluppo agile è la realizzazione delle User Stories che rappresentano le funzionalità che l'utente vuole realizzare con il software richiesto. Ogni user story è definita da:
		\begin{enumerate}
			\item descrizione del problema che è stato individuato;
			\item minuta delle conversazione tra gli stakeholder per discutere e comprendere il problema insieme;
			\item la strategia che il software utilizza per risolvere il problema.
		\end{enumerate}
	
		\item \textbf{Riunioni}:
	\begin{itemize}
		\item \textbf{Sprint planning}: pianificare il lavoro da svolgere durante lo sprint;
		\item \textbf{Sprint review}: riunione retrospettiva per verificare il lavoro svolto durante lo sprint;
		\item \textbf{Backlog refinement}: per aggiungere nuove User Stories o migliorare e/o modificare User Stories già create;
		\item \textbf{Riunioni giornaliere}: sostituite con comunicazioni telematiche dato la particolare situazione per verificare lo svolgimento del lavoro.
	\end{itemize}
\end{itemize}
%\subsection{Sprint di sviluppo}
% scrivere descrizione più accurata della metodologie agili applicate
