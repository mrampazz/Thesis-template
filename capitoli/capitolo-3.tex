% !TEX encoding = UTF-8
% !TEX TS-program = pdflatex
% !TEX root = ../tesi.tex

%**************************************************************
\chapter{Progettazione}
\label{cap:progettazione}
%**************************************************************

%\intro{Breve introduzione al capitolo}\\
% struttura maybe?
\section{Suddivisione dei componenti grafici}
In questa sezione verranno descritti i componenti grafici e i relativi Container Redux. I componenti grafici si occupano solamente di renderizzare i dati. I container Redux si occupano di mappare lo State di Redux ai props di ogni componente in modo che essi possano accedere solo alla porzione di stato di cui hanno bisogno.
\subsection{View}
\subsubsection{TableView}
Questo componente è stato realizzato per definire la struttura generale del componente e per suddividere in modo responsivo lo spazio in quattro quadranti. Come si può vedere dalla seguente immagine ho realizzato il componente utilizzando il layout CSS Grid. Questo mi ha permesso di gestire, in modo responsivo, la struttura del componente.

\subsubsection{TableHeaderView}
Questo componente rappresenta le dimensioni superiori della tabella pivot.
Dire che la struttura della tabella è stata realizzata con flex, table, etc.

\subsubsection{TableSidebarView}
Questo componente rappresenta le dimensioni laterali della tabella pivot.
Dire che la struttura della tabella è stata realizzata con flex, table, etc.

\subsubsection{TableBodyView}
Questo componente rappresenta i dati della tabella pivot
Dire che la struttura della tabella è stata realizzata con flex, table, etc.

\subsection{Container Redux}
\subsubsection{TableController}
\subsubsection{TableHeaderController}
\subsubsection{TableSidebarController}
\subsubsection{TableBodyController}

\section{Struttura dati dell'applicazione}

\subsection{Descrizione}
Per definire lo stato dell'applicazione sono state create delle 'data class' per gestire in modo ordinato lo stato del prodotto.
\subsubsection{TableState}
Lo stato del prodotto è definito da un singolo oggetto: 'data class TableState'.
\subsubsection{Cells}
\subsubsection{CellAction}
\subsubsection{BodyCells}



\section{Realizzazione del JSON parser}
Lo stato dell'applicazione viene popolato in seguito ad una richiesta ad API. Queste API ritornano risultati in formato JSON. Lavorando con Kotlin non è possibile avere la stessa libertà di accesso ad un JSON come in Javascript, quindi sono stati creati delle classi per eseguire la lettura del JSON e la sua traduzione in oggetti 'data class'. GRUPPO4 ha fornito una API che deve essere compatibili con questo componente quindi per ho realizzato un parser per trasformare un json in oggetti 'data class' e un adapter per trasformare gli oggetti data class nella data structure utilizzata nello stato dell'applicazione.

\subsection{Adapter: creavista}
Descrivi l'API\\
Descrivi il parser\\
La realizzazione del parse e del successivo adapter è stato uno delle difficoltà più grandi che ho superato. La struttura del JSON faceva cagare. Per realizzare un datastructure decente ho preso i dati dal json e li ho trasformati in una rappresentazione ad albero. Nel seguente modo:
ESEMPIO FOTO OR WHATEVER.
Dopodichè ho trasformato la rappresentazione ad albero degli oggetti in List che costituite nel seguente modo: ESEMPIO. IN particolare per quanto riguarda la dimensione delle righe ho suddiviso questa list in tante liste quante sono le dimensioni delle righe. Mentre per le coonne ho suddiviso la List in tante liste quante sono le dimensioni delle colonne.

Descrivi le funzioni\\


%**************************************************************
%\section{Introduzione al progetto}

%**************************************************************
%\section{Analisi preventiva dei rischi}

%Durante la fase di analisi iniziale sono stati individuati alcuni possibili rischi a cui si potrà andare incontro.
%Si è quindi proceduto a elaborare delle possibili soluzioni per far fronte a tali rischi.\\

%\begin{risk}{Performance del simulatore hardware}
%    \riskdescription{le performance del simulatore hardware e la comunicazione con questo potrebbero risultare lenti o non abbastanza buoni da causare il fallimento dei test}
%    \risksolution{coinvolgimento del responsabile a capo del progetto relativo il simulatore hardware}
 %   \label{risk:hardware-simulator} 
%\end{risk}

%**************************************************************
%\section{Requisiti e obiettivi}


%**************************************************************
%\section{Pianificazione}