% !TEX encoding = UTF-8
% !TEX TS-program = pdflatex
% !TEX root = ../tesi.tex

%**************************************************************
\chapter{Analisi dei requisiti}
\label{cap:analisi-requisiti}
%**************************************************************
\section{Definizione delle User Stories}
Per definire in modo semplice i requisiti del progetto sono state realizzate le \emph{User Stories}. Nell'ambito di questo tirocinio sono state definite mediante la seguente struttura.
\begin{longtable} {
		|>{}p{10mm}| 
		|>{}p{70mm}|
		|>{}p{15mm}|
		|>{}p{25mm}|
		>{}p{0mm}}
	\hline
	\textbf{Id} & \textbf{Descrizione} & \textbf{Priorità} & \textbf{Implementato} \\ \hline
	US1.1 & Descrizione dell'user story & A & SI \\ \hline
	\hline
	\caption{Esempio tabella User Story}
\end{longtable}
\noindent
Per ogni descrizione di un user story si possono identificare le seguenti informazioni:
\begin{itemize}
	\item \textbf{Ruolo}: definisce il tipo di utente;
	\item \textbf{Obiettivo}: definisce di che cosa ha bisogno l'utente;
	\item \textbf{Beneficio}: definisce i vantaggi che porta all'utente.
\end{itemize} 
\noindent
La sinteticità e la facilità nel definire le user story porta a vantaggi nella comunicazione tra il team di sviluppo e il cliente, rende più semplice l'aggiornamento dei requisiti e i costi di scrittura e manutenzione delle user stories sono molto bassi.

\section{Definizione del Product Backlog}
Uno dei primi obiettivi del progetto è stato quello di definire il Product Backlog, cioè i requisiti del prodotto. Per ognuna delle user story precedentemente scritta è stata assegnata una priorità seguendo la seguente legenda:
\begin{longtable} {
		|>{\centering}p{10mm}| 
		|>{}p{25mm}|
		|>{}p{85mm}|
		>{}p{0mm}}
	\hline
	\textbf{A} & \textit{Priorità alta}  & funzionalità necessarie per il corretto funzionamento dell'applicazione \\ \hline
	\textbf{M} & \textit{Priorità media} & funzionalità che migliorano il prodotto \\ \hline
	\textbf{B} & \textit{Priorità bassa} & funzionalità non necessarie per il corretto funzionamento dell'applicazione \\ \hline
	\hline
	\caption{Tabella priorità User Story}
\end{longtable}
\noindent
Questo mi ha permesso di categorizzare le funzionalità principali da quelle opzionali. Ho quindi popolato il Product Backlog per ordine di \textit{priorità}. In questo modo la suddivisione dei requisiti per sprint, mediante le riunioni di \textit{Sprint Planning}, è stata immediata. \\
Le funzionalità principali sono state pianificate per i primi due sprint così da avere, già a partire dal terzo sprint, un prodotto funzionante. Dato che ogni sprint di sviluppo prevede una riunione di \emph{backlog refinement} verrano elencate tutte le iterazioni del \emph{Product Backlog}.

\section{Product Backlog}
\begin{longtable} {
		|>{}p{10mm}| 
		|>{}p{70mm}|
		|>{}p{15mm}|
		|>{}p{25mm}|
		>{}p{0mm}}
	\hline
	\textbf{Id} & \textbf{Descrizione} & \textbf{Priorità} & \textbf{Implementato} \\ \hline
	US1 & Come \textbf{utente} voglio poter visualizzare i miei dati e le dimensioni relative ai dati & A & SI \\ \hline
	
	US2 & Come \textbf{utente} voglio avere una interfaccia non ostruttiva & M & SI \\ \hline
	
	US3 & Come \textbf{utente} voglio poter utilizzare questa applicazione web dal mio PC & A & SI \\ \hline
	
	US4 & Come \textbf{utente} voglio poter utilizzare questa applicazione web dal mio tablet & A & SI \\ \hline
	
	US5 & Come \textbf{utente} voglio poter utilizzare questa applicazione web dal mio telefono & A & SI \\ \hline
	
	US6 & Come \textbf{utente} voglio avere un caricamento veloce & M & SI \\ \hline
	
	US7 & Come \textbf{utente} voglio poter esplorare liberamente i dati & A & SI \\ \hline
	
\end{longtable}
\newpage
\section{Requisiti individuati dal Product Backlog}
\begin{longtable} {
		|>{}p{10mm}| 
		|>{}p{60mm}|
		|>{}p{15mm}|
		|>{}p{15mm}|
		|>{}p{15mm}|
		>{}p{0mm}}
	\hline
	\textbf{Id} & \textbf{Descrizione} & \textbf{Tipo} & \textbf{Impl.} & \textbf{User Story} \\ \hline
	
	R1.0   & \textbf{Definizione dello stato dell'applicazione} & O & SI & US1.1\\ \hline
	R1.0.1 & Sviluppo TableState        & O & SI & US1.1\\ \hline
	R1.0.2 & Sviluppo NodeDimensions    & O & SI & US1.1\\ \hline
	R1.0.3 & Sviluppo BodyCells         & O & SI & US1.1\\ \hline
	R1.0.4 & Sviluppo HeaderAction      & O & SI & US1.1\\ \hline
	R1.0.5 & Sviluppo NodeActionType    & O & SI & US1.1\\ \hline
	R1.1   & Sviluppo architettura Redux & O & SI & - \\ \hline
	R1.1.1 & Sviluppo TableStateSlice    & O & SI & - \\ \hline
	R1.1.2 & Sviluppo di Thunk & O & SI & - \\ \hline
	R1.1.3 & Sviluppo delle Actions & O & SI & - \\ \hline
	R1.1.4 & Sviluppo dei Reducers & O & SI & - \\ \hline
	
	R2.0   & \textbf{Definizione e pianificazione dei componenti} & O & SI & US1.1\\ \hline
	R2.1   & Sviluppo dei componenti React 		  & O & SI & US1, US2, US3, US4, US5 \\ \hline
	R2.1.1 & Sviluppo di TableView                & O & SI & -     \\ \hline
	R2.1.2 & Sviluppo di TableHeaderView          & O & SI & -     \\ \hline
	R2.1.3 & Sviluppo di TableSidebarView         & O & SI & -     \\ \hline
	R2.1.4 & Sviluppo di TableBodyView            & O & SI & -     \\ \hline
	R2.2   & Sviluppo dei container Redux         & O & SI & -     \\ \hline
	R2.2.2 & Sviluppo di TableController          & O & SI & -     \\ \hline
	R2.2.3 & Sviluppo di TableHeaderController    & O & SI & -     \\ \hline
	R2.2.4 & Sviluppo di TableSidebarController   & O & SI & -     \\ \hline
	R2.2.5 & Sviluppo di TableBodyController      & O & SI & -     \\ \hline

	R3.0   & \textbf{Sviluppo parser per JSON dell'API}         & O & SI & -     \\ \hline
	R3.1   & Sviluppo data class @Serializable 	   & O & SI & -     \\ \hline
	R3.1.1 & Sviluppo Gruppo4JSON & O & SI & - \\ \hline
	R3.1.2 & Sviluppo Gruppo4Data & O & SI & - \\ \hline
	R3.1.3 & Sviluppo Gruppo4Filter & O & SI & - \\ \hline
	R3.1.4 & Sviluppo Gruppo4Actions & O & SI & - \\ \hline
	R3.1.5 & Sviluppo Gruppo4Node & O & SI & - \\ \hline
	R3.2   & \textbf{Sviluppo adapter da JSON a TableState} & O & SI & -     \\ \hline
	R3.2.1   & Sviluppo funzione convertListOfGruppo4Node() & O & SI & -     \\ \hline
	R3.2.2   & Sviluppo funzione convertTree() & O & SI & -     \\ \hline
	R3.2.3   & Sviluppo funzione convertListOfGruppo4Actions() & O & SI & -     \\ \hline
	R3.2.4   & Sviluppo funzione convertCells() & O & SI & -     \\ \hline
	
	R4.0 & \textbf{Sviluppo funzioni per effettuare richieste HTTP}  & O & SI & -     \\ \hline
	R4.1   & Sviluppo funzione fetch() & O & SI & -     \\ \hline
	R4.2   & Sviluppo funzione sendAction() & O & SI & -     \\ \hline
	
	R5.0 & \textbf{Sviluppo caricamento parziale}  & O & SI & -     \\ \hline
	R5.1   & Sviluppo componente infinteScroller & O & SI & -     \\ \hline
	R5.2   & Sviluppo funzioni per aggiornare struttura ad albero & O & SI & -     \\ \hline
	\hline
\end{longtable}