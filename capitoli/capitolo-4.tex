% !TEX encoding = UTF-8
% !TEX TS-program = pdflatex
% !TEX root = ../tesi.tex

%**************************************************************
\chapter{Analisi dei requisiti}
\label{cap:analisi-requisiti}
%**************************************************************
\subsection{Definizione delle User Stories}
Per prima cosa il team di sviluppo si è occupato di realizzare le User Stories. Per definire un singolo elemento è stata utilizzata la seguente struttura:
\begin{longtable} {
		|>{}p{10mm}| 
		|>{}p{70mm}|
		|>{}p{15mm}|
		|>{}p{25mm}|
		>{}p{0mm}}
	\hline
	\textbf{Id} & \textbf{Descrizione} & \textbf{Priorità} & \textbf{Implementato} \\ \hline
	US1.1 & Descrizione dell'user story & A & SI \\ \hline
	\hline
	\caption{Esempio tabella User Story}
\end{longtable}
\noindent
Per ogni descrizione di un user story si possono identificare le seguenti informazioni:
\begin{itemize}
	\item \textbf{Ruolo}: definisce il tipo di utente;
	\item \textbf{Obiettivo}: definisce di che cosa ha bisogno l'utente;
	\item \textbf{Beneficio}: definisce i vantaggi che porta all'utente.
\end{itemize} 
\noindent
La sinteticità e la facilità nel definire le user story porta a vantaggi nella comunicazione tra il team di sviluppo e il cliente, rende più semplice l'aggiornamento dei requisiti e i costi di scrittura e manutenzione delle user stories sono molto bassi.

\subsection{Definizione del Product Backlog}
Uno dei primi obiettivi del progetto posti dal team di sviluppo è stato quello di definire il Product Backlog, cioè i requisiti del prodotto. Per ognuna delle user story precedentemente scritta è stata assegnata una priorità seguendo la seguente legenda:
\begin{longtable} {
		|>{\centering}p{10mm}| 
		|>{}p{25mm}|
		|>{}p{85mm}|
		>{}p{0mm}}
	\hline
	\textbf{A} & \textit{Priorità alta}  & funzionalità necessarie per il corretto funzionamento dell'applicazione \\ \hline
	\textbf{M} & \textit{Priorità media} & funzionalità che migliorano il prodotto \\ \hline
	\textbf{B} & \textit{Priorità bassa} & funzionalità non necessarie per il corretto funzionamento dell'applicazione \\ \hline
	\hline
	\caption{Tabella priorità User Story}
\end{longtable}
\noindent
Questo ci ha permesso di categorizzare le funzionalità principali del componente d'interfaccia grafico da quelle opzionali. Abbiamo quindi popolato il Product Backlog per ordine di \textit{priorità}, in questo modo la suddivisione delle user story per sprint, mediante le riunioni di \textit{Sprint Planning}, è stata chiara e veloce. \\
Infatti abbiamo concentrato le funzionalità principali da implementare nei primi due sprint così da avere già a partire dal terzo sprint una prodotto con le funzionalità principali già implementate.


\section{Product Backlog}
\begin{longtable} {
		|>{}p{10mm}| 
		|>{}p{70mm}|
		|>{}p{15mm}|
		|>{}p{25mm}|
		>{}p{0mm}}
	\hline
	\textbf{Id} & \textbf{Descrizione} & \textbf{Priorità} & \textbf{Implementato} \\ \hline
	US1.1 & Come Cliente voglio poter visualizzare i miei dati e le dimensioni relative ai dati & A & SI \\ \hline
	US1.2 & Come Cliente voglio poter utilizzare questa applicazione web dal mio PC, dal mio telefono e dal mio Tablet & A & SI \\ \hline
	US1.3 & Come Cliente voglio poter visualizzare solo i dati senza le dimensioni relative ad essi & M & SI \\ \hline
	US2.1 & Come Cliente voglio poter utilizzare l'API precedente di Creavista & A & SI \\ \hline
	US2.2 & Come Cliente voglio poter utilizzare la nuova API di Creavista & A & SI \\ \hline
	US2.3 & Come Cliente voglio avere un caricamento veloce & M & SI \\ \hline
	US3.1 & Come cliente voglio poter salvare la mia configurazione & M & SI \\ \hline
	US3.2 & Come cliente voglio poter esportare i miei dati & B & SI \\ \hline
	US3.3 & Come cliente voglio poter decidere quali filtri voglio utilizzare per ogni dimensione & A & SI \\ \hline
	US3.4 & Come cliente voglio poter modificare i filtri che sto utilizzando & A & SI \\ \hline
	\hline
\end{longtable}
\newpage
\section{Requisiti individuati dal Product Backlog}
\begin{longtable} {
		|>{}p{10mm}| 
		|>{}p{60mm}|
		|>{}p{15mm}|
		|>{}p{15mm}|
		|>{}p{15mm}|
		>{}p{0mm}}
	\hline
	\textbf{Id} & \textbf{Descrizione} & \textbf{Tipo} & \textbf{Impl.} & \textbf{User Story} \\ \hline
	R1.0   & Definizione e pianificazione dei componenti & O & SI & US1.1\\ \hline
	
	R1.1   & Sviluppo dei componenti React visivi & O & SI & US1.1 \\ \hline
	R1.1.1 & Sviluppo di TableView                & O & SI & -     \\ \hline
	R1.1.2 & Sviluppo di TableHeaderView          & O & SI & -     \\ \hline
	R1.1.3 & Sviluppo di TableSidebarView         & O & SI & -     \\ \hline
	R1.1.4 & Sviluppo di TableBodyView            & O & SI & -     \\ \hline
	
	R1.2   & Sviluppo dei container Redux       & O & SI & US1.1 \\ \hline
	R1.1.1 & Sviluppo di TableController        & O & SI & -     \\ \hline
	R1.1.2 & Sviluppo di TableHeaderController  & O & SI & -     \\ \hline
	R1.1.3 & Sviluppo di TableSidebarController & O & SI & -     \\ \hline
	R1.1.4 & Sviluppo di TableBodyController    & O & SI & -     \\ \hline
	
	R2.0   & Sviluppo dei JSON parser          & O & NO & US2.1 \\ \hline
	
	R2.1   & Sviluppo parser per Creavista     & O & SI & -     \\ \hline
	R2.1.1 & Sviluppo data structure in Kotlin & O & SI & -     \\ \hline
	R2.1.2 & Sviluppo adapter                  & O & SI & -     \\ \hline
	
	R2.2   & Sviluppo parser per API nuova     & O & NO & US2.2 \\ \hline
	R2.2.1 & Sviluppo data structure in Kotlin & O & NO & -     \\ \hline
	R2.2.2 & Sviluppo adapter                  & O & NO & -     \\ \hline
	
	R3.0 & Sviluppo di un sistema di caricamento automatico                   & O & NO & US2.3 \\ \hline
	R3.1 & Sviluppo di InfiniteScroller per gestire il caricamento automatico & O & SI & - \\ \hline
	\hline
\end{longtable}