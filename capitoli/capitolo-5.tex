% !TEX encoding = UTF-8
% !TEX TS-program = pdflatex
% !TEX root = ../tesi.tex

%**************************************************************
\chapter{Tecnologie e strumenti}
\label{cap:tecnologie-strumenti}
%**************************************************************

%\intro{Breve introduzione al capitolo}\\

%**************************************************************
\section{Tecnologie}
\label{sec:tecnologie}
INSERIRE L'IDEA DI COME LE TECNOLOGIE SONO STATE UTILIZZATE NEL PROGETTO

\subsubsection*{Kotlin}
Kotlin è un linguaggio tipizzato, realizzato da JetBrains, utilizzato da molti sviluppatori per il fatto che il codice è conciso, sicuro e permette di lavorare utilizzando libreria per la JVM, Android e il browser.

\subsubsection*{React}
React è una libreria che presenta principalmente due caratteristiche:
\begin{itemize}
	\item l'uso del Virtual DOM;
	\item realizzazione di componenti migliorare la reutilizzazione del codice.
\end{itemize}

\subsubsection*{Kotlin Wrappers}
\paragraph{kotlin-react} \mbox{} \\
\paragraph{kotlin-redux} \mbox{} \\
\paragraph{kotlin-react-redux} \mbox{} \\


\section{Strumenti}
\label{sec:strumenti}

Di seguito viene data una descrizione delle tecnologie che sono stati utilizzati durante il tirocinio.


% parlare magari di come si è scritto il codice magari

%**************************************************************
%\section{Ciclo di vita del software}
%\label{sec:ciclo-vita-software}

%**************************************************************
%\section{Progettazione}
%\label{sec:progettazione}

%\subsubsection{Namespace 1} %**************************
%Descrizione namespace 1.

%\begin{namespacedesc}
%    \classdesc{Classe 1}{Descrizione classe 1}
%    \classdesc{Classe 2}{Descrizione classe 2}
%\end{namespacedesc}


%**************************************************************
%\section{Design Pattern utilizzati}

%**************************************************************
%\section{Codifica}
