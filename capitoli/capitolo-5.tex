% !TEX encoding = UTF-8
% !TEX TS-program = pdflatex
% !TEX root = ../tesi.tex

%**************************************************************
\chapter{Tecnologie}
\label{cap:tecnologie-strumenti}
%**************************************************************

%\intro{Breve introduzione al capitolo}\\

%**************************************************************
\section{Kotlin}
Kotlin è un linguaggio tipizzato, realizzato da JetBrains, utilizzato da molti sviluppatori per il fatto che il codice è conciso, sicuro e permette di lavorare utilizzando librerie per la JVM, Android e il browser.

\section{React}
Libreria utilizzata per la realizzazione dei componenti grafici dell'applicazione.

\section{Kotlin Wrappers}
Durante la codifica del progetto sono state utilizzati alcuni wrapper forniti da JetBrains per lo sviluppo web di React, Redux e React-Redux.

\subsection{kotlin-react}
Wrapper utilizzato per l'utilizzo di funzioni che permettono la codifica di componenti React.

\subsection{kotlin-redux}
Wrapper utilizzato per la realizzazione dell'architettura Redux utilizzata nell'applicazione per la gestione dello stato.

\subsection{kotlin-react-redux}
Wrapper utilizzato per la realizzazione del collegamento tra i componenti React e lo stato di Redux.

\section{Redux}
Libreria utilizzata per la realizzazione dello stato dell'applicazione e della sua gestione


% parlare magari di come si è scritto il codice magari

%**************************************************************
%\section{Ciclo di vita del software}
%\label{sec:ciclo-vita-software}

%**************************************************************
%\section{Progettazione}
%\label{sec:progettazione}

%\subsubsection{Namespace 1} %**************************
%Descrizione namespace 1.

%\begin{namespacedesc}
%    \classdesc{Classe 1}{Descrizione classe 1}
%    \classdesc{Classe 2}{Descrizione classe 2}
%\end{namespacedesc}


%**************************************************************
%\section{Design Pattern utilizzati}

%**************************************************************
%\section{Codifica}
