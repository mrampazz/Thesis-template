% !TEX encoding = UTF-8
% !TEX TS-program = pdflatex
% !TEX root = ../tesi.tex

%**************************************************************
\chapter{Tecnologie e strumenti}
\label{cap:tecnologie-strumenti}
%**************************************************************

%\intro{Breve introduzione al capitolo}\\

%**************************************************************
\section{Tecnologie}
\label{sec:tecnologie}
INSERIRE L'IDEA DI COME LE TECNOLOGIE SONO STATE UTILIZZATE NEL PROGETTO

\subsubsection*{Kotlin}
Kotlin è un linguaggio tipizzato, realizzato da JetBrains, utilizzato da molti sviluppatori per il fatto che il codice è conciso, sicuro e permette di lavorare utilizzando libreria per la JVM, Android e il browser.

\subsubsection*{React}
React è una libreria che presenta principalmente due caratteristiche:
\begin{itemize}
	\item l'uso del Virtual DOM;
	\item realizzazione di componenti migliorare la reutilizzazione del codice.
\end{itemize}

\subsubsection*{Kotlin Wrappers}
\paragraph{kotlin-react} \mbox{} \\
\paragraph{kotlin-redux} \mbox{} \\
\paragraph{kotlin-react-redux} \mbox{} \\


\section{Strumenti}
\label{sec:strumenti}

Di seguito viene data una descrizione delle tecnologie che sono stati utilizzati durante il tirocinio.

\subsection{Versionamento della soluzione}
\subsubsection*{Git}
Git è un VCS (Version Control System) distribuito che permette di tenere traccia delle modifiche in un prodotto software e di organizzare la codifica del prodotto.
\subsubsection*{GitLab}
Strumento web che permette di implementare un DevOps lifecycle che fornisce una gestione di repository git, un ITS (Issue Tracking System) e altri strumenti quali la "Continuous integration" e "Continuous deployement".

\subsection{Ambiente di sviluppo locale}
\subsubsection*{IntelliJ IDEA Community Edition}
IntelliJ IDEA Community Edition è una IDE realizzata da JetBrains che fornisce funzionalità di supporto per lo sviluppo di molti linguaggi, specialmente Kotlin.
\subsubsection*{Gradle}
Gradle è uno strumento di "Build automation" per molti linguaggi tra cui Kotlin e Java. E' stato usato per la gestione e l'installazione delle dipendenze.

\subsection{Organizzazione del lavoro}
\subsubsection*{Trello}
Per l'organizzazione del lavoro, in particolare per la gestione dei macro obiettivi di ogni sprint, ho utilizzato Trello che fornisce un'interfaccia Kanban.

% parlare magari di come si è scritto il codice magari

%**************************************************************
%\section{Ciclo di vita del software}
%\label{sec:ciclo-vita-software}

%**************************************************************
%\section{Progettazione}
%\label{sec:progettazione}

%\subsubsection{Namespace 1} %**************************
%Descrizione namespace 1.

%\begin{namespacedesc}
%    \classdesc{Classe 1}{Descrizione classe 1}
%    \classdesc{Classe 2}{Descrizione classe 2}
%\end{namespacedesc}


%**************************************************************
%\section{Design Pattern utilizzati}

%**************************************************************
%\section{Codifica}
