% !TEX encoding = UTF-8
% !TEX TS-program = pdflatex
% !TEX root = ../tesi.tex

%**************************************************************
\chapter{Conclusioni}
\label{cap:conclusioni}
%**************************************************************
\section{Raggiungimento degli obiettivi}
Gli obiettivi del progetto sono stati raggiunti quasi completamente. Tutti i requisiti richiesti dall'azienda sono stati soddisfatti tranne il caricamento parziale di sezioni della tabella a causa dei problemi riscontrati durante la codifica dello \emph{sprint}: \textbf{S3}.

%**************************************************************
\section{Conoscenze acquisite e valutazioni personali}
In questa sezione verranno identificate le conoscenze che ho acquisito per ogni metodologia e tecnologia utilizzate nell'ambito del progetto e la loro efficacia da un punto di vista personale.

\subsection{Metodologia agile}
La metodologia agile simile a SCRUM utilizzata dal team di sviluppo mi ha permesso di identificare dal punto di vista pratico le componenti necessarie per lavorare in un team di sviluppo che pratica una metodologia agile. In particolare ho imparato a gestire in modo più efficiente la codifica dato le scadenze presenti in ogni \emph{sprint}. 

\subsubsection*{Efficacia}
Considero l'utilizzo di una metodologia agile in un progetto software molto utile dato che il suo utilizzo permette di risolvere problemi in modo efficiente. Inoltre grazie alle sue riunioni favorisce una comunicazione costante tra i membri del progetto. La suddivisione in sprint di sviluppo, se pianificata correttamente, aiuta a mantenere il lavoro concentrato su un numero ristretto di funzionalità di un prodotto e questo può aiutare a suddividerlo in piccoli incrementi. Il vantaggio che ho notato maggiormente durante la codifica in \emph{sprint} è stato il fatto che la metodologia agile poneva come obiettivo la realizzazione delle funzionalità più importanti nei primi \emph{sprint}, questo mi ha permesso di avere una visione completa dell'applicazione già dai primi \emph{sprint}.

\subsection{Kotlin}
Durante lo studio iniziale e la codifica del componente d'interfaccia grafica ho maturato la mia conoscenza del linguaggio di programmazione Kotlin. Considero questa conoscenza molto importante dato che Kotlin è uno dei linguaggi di programmazione in più rapida crescita negli ultimi anni. Inoltre in quanto è possibile poter sviluppare prodotti per molte piattaforme penso che questo progetto mi abbia fornito le basi per arricchire le mie abilità su altri campi oltre che naturalmente lo sviluppo di applicazioni web.

\subsubsection*{Efficacia}
Lo sviluppo in Kotlin di interfacce utente per il web è una soluzione valida ma dal mio punto di vista deve maturare ancora. La documentazione per Kotlin riguardo lo sviluppo web è scarsa e gli esempi disponibili online sono molto limitati. Inoltre i \emph{wrapper} forniti da Jetbrains sono ancora incompleti per quanto riguarda le funzionalità di React e Redux. Per questi motivi penso che Kotlin, in questo momento, non è una soluzione adatta per la realizzazione di interfacce web. Infatti i vantaggi che porta Kotlin non superano, a mio parere, la comodità di JavaScript o TypeScript nello sviluppo di applicazioni web.

\subsection{React e Redux}
La libreria React è utilizzata da molte aziende per la realizzazione di interfacce utente per applicazioni web. In passato avevo già lavorato con React in alcuni progetti personali tuttavia, in questo tirocinio, ho ampliato e soprattutto affinato le mie conoscenze di questa libreria. \\
Lo stesso vale per Redux, infatti grazie allo studio e alla codifica di componenti React-Redux considero di aver migliorato e ampliato le mie conoscenze nello sviluppo web di applicazioni scalabili e manutenibili.

\subsubsection*{Efficacia}
Considero questo due librerie molto utili, specialmente Redux dato che aiuta a gestire lo stato dell'appplicazione in modo prevedibile. Questo permette di realizzare applicazioni scalabili, un fattore molto importante specialmente nello sviluppo di applicazioni web. La scrittura di componenti React permette di riutilizzare codice, inoltre la divisione per componenti aiuta a organizzare meglio la struttura di una interfaccia utente. Un altro fattore positivo di queste due libreria è il loro uso combinato. Infatti forniscono funzioni che permettono di integrare molto facilmente la gestione dello stato di Redux ai componenti di React.

\subsection{Programmazione funzionale}
Lo studio delle \emph{first class function}s e del loro utilizzo mi hanno permesso di entrare nel mondo della programmazione funzionale. Insieme a Kotlin e alla metodologia agile considero che queste nozioni di programmazione funzionale mi permetterranno in futuro di velocizzare la codifica e la progettazione di nuove applicazioni.

\subsubsection*{Efficacia}
L'utilizzo delle \emph{higher order function} ha moltissimi vantaggi per quanto riguarda la codifica di una applicazione. I vantaggi che ho identificato sono stati: aumento della velocità della codifica, aumento della leggibilità del codice durante la verifica (a causa della natura dichiarativa) e una scrittura più chiara e concisa di funzioni. \\
Tuttavia ho notato che è presente il rischio di sviluppare insiemi di funzioni molto complicate che possono causare confusione se vengono lette da un altro programmatore.

\subsection{Lavorare da remoto e considerazioni finali}
Dato le circostanze sociali ho lavorato principalmente da remoto, questo ha portato ad alcune difficoltà per quanto riguarda la comunicazione anche se la l'applicazione della metodologia agile ha comunque aiutato. \\ 
Considero questa esperienza lavorativa molto positiva in quanto ho avuto la possibilità di lavorare con persone qualificate nel settore informatico in cui sono più interessato: lo sviluppo web. 
