% !TEX encoding = UTF-8
% !TEX TS-program = pdflatex
% !TEX root = ../tesi.tex

%**************************************************************
\chapter{Conclusioni}
\label{cap:conclusioni}
%**************************************************************

%**************************************************************
\section{Problemi riscontrati}
I principali problemi che sono stati riscontrati durante al progetto sono stati dovuti principalmente alla mancanaza di documentazione ed esempi nell'utilizzo combinato di Kotlin, React e Redux. Rallentamenti per il continuo cambiamento della struttura dell'API da parte dell'azienda e la relativa realizzazione dell'adapter.

%**************************************************************
\section{Raggiungimento degli obiettivi}
Gli obiettivi del progetto sono stati raggiunti quasi completamente. L'unico obiettivo che non è stato completato è stato il caricamento continuo allo scroll di un utente dato il rallentamento durante lo sviluppo del parser.

%**************************************************************
\section{Conoscenze acquisite}
\subsection{Metodologia agile}
\subsection{Kotlin}
\subsection{React e Redux}
\subsection{Programmazione funzionale}
\subsection{Lavorare in un team di sviluppo}
\subsection{Lavorare da remoto}

%**************************************************************
\section{Valutazione personale}
\subsection{Effettività delle metodologie agili}
\subsection{Effettività di Kotlin per la realizzazione di UI per il web}
Lo sviluppo in Kotlin di interfacce utente per il web funziona ma secondo me deve maturare ancora perchè ci sono ancora molte limitazioni / problemi che su typescript/javascript non ci sono. Ci sono molti vantaggi nell'utilizzare kotlin ma in questo momento gli svantaggi superano i vantaggi.
\subsection{Effettività di React e Redux}
Molto utili, specialmente Redux che aiuta a gestire lo stato dell'appplicazione in modo molto pulito e permette di rendere molto più scalabili applicazioni web che di solito sono considerate molto poco scalabili.
\subsection{Effettività della programmazione funzionale}
Molto utile perchè, meno codice, più conciso, più veloce da scrivere.