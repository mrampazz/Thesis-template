% !TEX encoding = UTF-8
% !TEX TS-program = pdflatex
% !TEX root = ../tesi.tex

%**************************************************************
\chapter{Conclusioni}
\label{cap:conclusioni}
%**************************************************************
\section{Raggiungimento degli obiettivi}
Gli obiettivi del progetto sono stati raggiunti quasi completamente. Tutti i requisiti richiesti dall'azienda sono stati soddisfatti tranne il caricamento continuo allo scroll di un utente dato i problemi riscontrati durante la codifica dello sprint numero 3.

%**************************************************************
\section{Conoscenze acquisite e valutazioni personali}
In questa sezione verranno identificate le conoscenze che ho acquisito per ogni metodologia e tecnologia utilizzate nell'ambito del progetto e la loro efficacia da un punto di vista personale.

\subsection{Metodologia agile}
La metodologia agile simile a SCRUM utilizzata dal team di sviluppo mi ha permesso di identificare dal punto di vista pratico le componenti necessarie a lavorare in un team di sviluppo che pratica una metodologia agile.

\subsubsection*{Efficacia}
Consider l'utilizzo di una metodologia agile in un progetto software molto utile dato che il suo utilizzo permette di risolvere problemi in modo efficiente. Inoltre grazie alle sue riunioni favorisce un ambiente ricco di comunicazione che considero molto importante. La suddivisione in sprint di sviluppo, se pianificato correttamente, aiuta a mantenere il lavoro concentrato su un numero ristretto di funzionalità di un progetto e questo può aiutare per suddividere il lavoro in piccoli incrementi. Il vantaggio che ho notato maggiormente durante la codifica in sprint è stato il fatto che, dato che la struttura della metodologia agile poneva come obiettivo la realizzazione delle funzionalità più importanti nei primi sprint, questo permette di avere una visione completa di un'applicazione già dai primi sprint.

\subsection{Kotlin}
Durante lo studio iniziale e la codifica del componente d'interfaccia grafica mi ha permesso di arricchire la mia conoscenza del linguaggio Kotlin. Considero questa conoscenza acquisita molto importante dato che Kotlin è utilizzato da molte aziende famose tra cui: ... . Inoltre in quanto è un linguaggio che permette di realizzare molti prodotti (api, web app, applicazioni android native, etc..) penso che questo stage mi abbia fornito le basi per espandere le mie abilità su altri campi oltre che naturalmente lo sviluppo di applicazioni web.

\subsubsection*{Efficacia}
Lo sviluppo in Kotlin di interfacce utente per il web è una soluzione valida ma secondo me è una tecnologia che deve maturare ancora per quanto riguarda lo sviluppo di applicazioni web. Infatti la documentazione sull'argomento è scarsa e gli esempi disponibili online sono molto limitati.

\subsection{React e Redux}
La libreria React è utilizzata molto per la realizzazione di applicazioni web da moltissime aziende. In passato avevo già lavorato con React in progetti personali tuttavia, in questo progetto, ho ampliato e soprattutto affinato le mie conoscenze di questa libreria. Lo stesso vale per Redux, infatti grazie allo studio e alla codifica di componenti React-Redux considero di aver migliorato e ampliato le mie conoscenze nello sviluppo web di applicazioni scalabili e manutenibili.

\subsubsection*{Efficacia}
Molto utili, specialmente Redux che aiuta a gestire lo stato dell'appplicazione in modo molto metodico e separato dall'interfaccia grafica. Questo permette di realizzare applicazioni scalabili, un fattore molto importante specialmente nello sviluppo di applicazioni web. La scrittura di componenti React permette di riutilizzare codice, inoltre la divisione per componenti aiuta a organizzare meglio la struttura di una interfaccia utente. 

\subsection{Programmazione funzionale}
Lo studio delle first class functions e del loro utilizzo mi hanno permesso di entrare nel mondo della programmazione funzionale. Insieme a Kotlin e alla metodologia agile considero che queste nozioni di programmazione funzionale mi permetterranno in futuro di velocizzare la codifica e la progettazione di nuove applicazioni.
\subsubsection*{Efficacia}
L'utilizzo delle \emph{higher order function} ha moltissimi vantaggi per quanto riguarda la codifica di una applicazione. I vantaggi che ho identificato sono stati: aumento della velocità della codifica, aumento della leggibilità del codice durante la verifica (a causa della natura dichiarativa) e una scrittura più chiara e concisa delle funzioni.

\subsection{Lavorare da remoto}
Dato le circostanze sociali ho lavorato principalmente da remoto, questo ha portato ad alcune difficoltà per quanto riguarda la comunicazione, tuttavia l'applicazione della metodologia agile ha aiutato.
