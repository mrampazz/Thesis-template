% !TEX encoding = UTF-8
% !TEX TS-program = pdflatex
% !TEX root = ../tesi.tex

%**************************************************************
\chapter{Codifica}
\label{cap:codifica}
%**************************************************************

La codifica del prodotto, come descritto nel capitolo 2 è stata suddivisa in sprint di sviluppo che corrispondono a circa 4-5 giorni lavorativi dove vengono implementati parte delle user stories indicate nel product backlog. Per ogni sprint verrano descritti lo Sprint Backlog e le Soluzioni che sono state implementate e i seguenti eventi: \textbf{Sprint Review} e \textbf{Backlog refinement}.
Infine verranno elencati i possibili ritardi, i problemi riscontrati e le soluzioni trovate. Ogni sprint è stato identificato da un codice univoco e da un titolo. In questo progetto gli sprint individuati sono stati:
\begin{itemize}
	\item \textbf{S1: Struttura dello stato e dell'architettura Redux};
	\item \textbf{S2: Componenti grafici e container Redux};
	\item \textbf{S3: Parser JSON e adapter};
	\item \textbf{S4: Caricamento parziale}.
\end{itemize}


\section{S1: Struttura dello stato e dell'architettura Redux}
\textbf{Durata:} \textit{5 giorni} \\
Nel primo sprint di sviluppo sono stati implementati i requisiti considerati più importanti per porre delle solide basi dell'applicazione web; in particolare la progettazione e codifica della struttura dello stato e l'architettura di Redux.
% maybe insert grafico

\subsection{Sprint Backlog}
In particolare sono stati implementati questi requisiti.
\begin{longtable} {
		|>{}p{10mm}| 
		|>{}p{90mm}|
		|>{}p{15mm}|
		|>{}p{15mm}|
		|>{}p{15mm}|
		>{}p{0mm}}
	\hline
	\textbf{Id} & \textbf{Descrizione} & \textbf{Tipo} \\ \hline
	R1.0 & Definizione dello stato dell'applicazione & O \\ \hline
	R1.0.1 & Sviluppo TableState        & O\\ \hline
	R1.0.2 & Sviluppo NodeDimensions    & O\\ \hline
	R1.0.3 & Sviluppo BodyCells         & O\\ \hline
	R1.0.4 & Sviluppo HeaderAction      & O\\ \hline
	R1.0.5 & Sviluppo NodeActionType    & O\\ \hline
	R1.1   & Definizione architettura Redux & O\\ \hline
	R1.1.1 & Sviluppo TableStateSlice    & O\\ \hline
	R1.1.2 & Sviluppo di Thunk & O\\ \hline
	R1.1.3 & Sviluppo delle Actions & O\\ \hline
	R1.1.4 & Sviluppo dei Reducers & O\\ \hline
\end{longtable}

\subsection{Soluzioni implementate}
\subsubsection{TableState}
L'entità \verb|TableState| è la \verb|data class| che contiene l'intero stato dell'applicazione. Al suo interno sono presenti tutte le informazioni necessarie per la corretta renderizzazione della tabella pivot.
\begin{lstlisting}[caption={TableState}, label={lst:tablestate}, language=Kotlin]
data class TableState(
val cols: ArrayList<ArrayList<DimensionsNode>>,
val rows: ArrayList<ArrayList<DimensionsNode>>,
val cells: ArrayList<ArrayList<BodyCells>>,
val rowAction: ArrayList<ArrayList<HeaderAction>>,
val colAction: ArrayList<ArrayList<HeaderAction>>
)
\end{lstlisting}

\subsubsection{DimensionsNode}
L'entità \verb|DimensionsNode| è la \verb|data class| che contiene le informazioni riguardanti una cella d'intestazione della tabella pivot.
\begin{lstlisting}[caption={DimensionsNode}, label={lst:dimensionsnode}, language=Kotlin]
data class DimensionsNode(
var id: String,
var label: String,
var level: Int? = null,
var childDepth: Int? = null,
var path: List<String>? = null,
var actionType: NodeActionType = NodeActionType.NULL,
var isChild: Boolean = false
)
\end{lstlisting}

\subsubsection{HeaderAction}
L'entità \verb|HeaderAction| è la \verb|data class| che contiene le informazioni riguardanti i pulsanti che si occupano di aprire intere colonne o righe di dimensioni nella TableActionUI.
\begin{lstlisting}[caption={HeaderAction}, label={lst:headeraction}, language=Kotlin]
data class HeaderAction(
val actionType: NodeActionType,
val dim: Int,
val depth: Int,
)
\end{lstlisting}

\subsubsection{NodeActionType}
L'entità \verb|HeaderAction| è una \verb|enum class| che definisce il tipo di azione che può essere eseguita da una cella.
\begin{lstlisting}[caption={NodeActionType}, label={lst:nodeactiontype}, language=Kotlin]
enum class NodeActionType(val type: String) {
EXPAND("E"),
COLLAPSE("C"),
NULL("")
}
\end{lstlisting}

\subsubsection{BodyCells}
L'entità \verb|BodyCells| è la \verb|data class| che contiene le informazioni riguardanti le celle che contengono i dati della tabella.
\begin{lstlisting}[caption={BodyCells}, label={lst:bodycells}, language=Kotlin]
data class BodyCells(
val value: Int,
val cPath: List<String>,
val rPath: List<String>
)
\end{lstlisting}

\subsubsection{Stato}
Lo stato gestito da Redux è equivalente alla struttura di \verb|TableState| con l'aggiunta del campo dato \verb|isLoading| così da avere un variabile per gestire la tabella durante il caricamento dall'API esterna di dati. Lo stato dell'applicazione è definito nel seguente modo:

\begin{lstlisting}[caption={State}, label={lst:bodycells}, language=Kotlin]
data class State(
val cols: ArrayList<ArrayList<DimensionsNode>> = ArrayList(),
val rows: ArrayList<ArrayList<DimensionsNode>> = ArrayList(),
val cells: ArrayList<ArrayList<BodyCells>> = ArrayList(),
val rowActions: ArrayList<ArrayList<HeaderAction>> = ArrayList(),
val colActions: ArrayList<ArrayList<HeaderAction>> = ArrayList(),
val isLoading: Boolean = false
)
\end{lstlisting}

\subsubsection{Thunk}
Dall'interfaccia \verb|RThunk| ho realizzato la classe \verb|InitState| che, mediante la funzione \verb|invoke| utilizza le coroutines di kotlin per riempire lo stato di \verb|TableStateSlice| con i dati ricevuti dall'API.
\begin{lstlisting}[caption={InitState}, label={lst:bodycells}, language=Kotlin]
class InitState : RThunk {
override fun invoke(
dispatch: (RAction) -> WrapperAction, 
getState: () -> AppState
): WrapperAction {
val mainScope = MainScope()
mainScope.launch {
val res : TableState = fetchCreavistaJson()
dispatch(TableStateSlice.UpdateRows(res.rows))
dispatch(TableStateSlice.UpdateCols(res.cols))
dispatch(TableStateSlice.UpdateCells(res.cells))
dispatch(TableStateSlice.UpdateRowActions(res.rowAction))
dispatch(TableStateSlice.UpdateColActions(res.colAction))
}
return nullAction
}
}
\end{lstlisting}

\subsubsection{Actions}
Le actions sono definiti come delle classi di tipo \verb|RAction| che vengono usati per innescare l'update dello stato.

\begin{lstlisting}[caption={Actions}, label={lst:bodycells}, language=Kotlin]
class UpdateCells(val cells: ArrayList<ArrayList<BodyCells>>): RAction
class UpdateRows(val rows: ArrayList<ArrayList<DimensionsNode>>): RAction
class UpdateCols(val cols: ArrayList<ArrayList<DimensionsNode>>): RAction
class SetIsLoading(val b: Boolean): RAction
class UpdateRowActions(val n: ArrayList<ArrayList<HeaderAction>>): RAction
class UpdateColActions(val n: ArrayList<ArrayList<HeaderAction>>): RAction
\end{lstlisting}


\subsubsection{Reducer}
Un \emph{reducer} è una \emph{funzione pura} che riceve come argomento un \verb|RAction| e ritorna una copia dello stato modificato. L'implementazione utilizzata in \verb|TableStateSlice| è la seguente:
\begin{lstlisting}[caption={Reducer}, label={lst:bodycells}, language=Kotlin]
fun reducer(state: State = State(), action: RAction) : State {
return when (action) {
is UpdateCells -> state.copy(cells = action.cells)
is UpdateRows -> state.copy(rows = action.rows)
is UpdateCols -> state.copy(cols = action.cols)
is SetIsLoading -> state.copy(isLoading = action.b)
is UpdateRowActions -> state.copy(rowActions = action.n)
is UpdateColActions -> state.copy(colActions = action.n)
else -> state
}
}
\end{lstlisting}

\subsection{Struttura del progetto}
\dirtree{%
	.1 kotlin.
		.2 redux.
			.3 slices.
				.4 TableStateSlice.kt.
			.3 thunks.
				.4 RThunk.kt.
				.4 InitState.kt.
			.3 state.
				.4 Index.kt.
		.2 entities.
			.3 StateEntities.kt.
}
\subsection{Sprint Review}
Niente da segnalare.

\subsection{Backlog refinement}
Il Product Backlog non è stato modificato.

\subsection{Problemi riscontrati}
Implementare correttamente l'architettura Redux è risultato difficoltoso a causa della mia inesperienza con la libreria e la mancanza di documentazione relativa a \verb|kotlin-redux|. Questi problemi non hanno però causato rallentamenti nell'implementazione degli altri requisiti dello Sprint Backlog.

\newpage

\section{S2: Componenti grafici e container Redux}
\textbf{Durata:} \textit{4 giorni} \\
Nel secondo sprint di sviluppo ho lavorato sull'interfaccia grafica e quindi i componenti React e i relativi container Redux.

\subsection{Sprint Backlog}
\begin{longtable} {
		|>{}p{10mm}| 
		|>{}p{90mm}|
		|>{}p{15mm}|
		|>{}p{15mm}|
		|>{}p{15mm}|
		>{}p{0mm}}
	\hline
R2.0   & \textbf{Definizione dei componenti} & O\\ \hline
R2.1   & Sviluppo dei componenti React 		  & O\\ \hline
R2.1.1 & Sviluppo di TableView                & O\\ \hline
R2.1.2 & Sviluppo di TableHeaderView          & O\\ \hline
R2.1.3 & Sviluppo di TableSidebarView         & O \\ \hline
R2.1.4 & Sviluppo di TableBodyView            & O\\ \hline
R2.2   & Sviluppo dei container Redux         & O  \\ \hline
R2.2.2 & Sviluppo di TableController          & O   \\ \hline
R2.2.3 & Sviluppo di TableHeaderController    & O      \\ \hline
R2.2.4 & Sviluppo di TableSidebarController   & O      \\ \hline
R2.2.5 & Sviluppo di TableBodyController      & O   \\ \hline
\end{longtable}
\subsection{Struttura del progetto}
\dirtree{%
	.1 kotlin.
	.2 redux.
	.3 slices.
	.4 TableStateSlice.kt.
	.3 thunks.
	.4 RThunk.kt.
	.4 InitState.kt.
	.3 state.
	.4 Index.kt.
	.3 container.
	.4 TableViewController.kt.
	.4 TableHeaderViewController.kt.
	.4 TableSidebarViewController.kt.
	.4 TableBodyViewController.kt.
	.4 TableActionUIController.kt.
	.4 TableLabelViewController.kt.
	.2 entities.
	.3 StateEntities.kt.
	.2 view.
	.3 TableView.kt.
	.3 TableHeaderView.kt.
	.3 TableSidebarView.kt.
	.3 TableBodyView.kt.
	.3 TableActionUI.kt.
	.3 TableLabelView.kt.
}

\subsection{Sprint Review}
Niente da segnalare.

\subsection{Backlog refinement}
Il Product Backlog non è stato modificato.

\subsection{Problemi riscontrati}
Niente da segnalare.
\newpage

\section{S3: Parser JSON e adapter}
\subsection{Sprint Backlog}
\subsection{Soluzioni implementate}
\subsubsection{Gestione delle richieste all'API}
Il primo passo per utilizzare dei dati reali all'interno della tabella pivot è stato quello di realizzare una funzione per effettuare richieste HTTP alla API di Gruppo4. Per farlo ho utilizzato l'implementazione in Kotlin di \verb|window.fetch|. La funzione risultante è la seguente:
\begin{lstlisting}[caption={Funzione fetch()}, label={lst:bodycells}, language=Kotlin]
suspend fun fetch() {
val res = window.fetch(url, RequestInit(
method = "GET",
credentials = RequestCredentials.Companion.INCLUDE,
headers = {
json("Accept" to "application/json")
json("Content-Type" to "application/json")
}))
.await()
.json()
.await()

// Codice identificativo dell'istanza della tabella pivot necessario
// per le chiamate successive
INSTANCE_KEY = (res as kotlin.js.Json)["InstanceKey"] as String?
}
\end{lstlisting}
La seguente funzione effettua una richiesta HTTP GET alla API di Gruppo4 la quale ritorna il JSON risultante e una chiave necessaria per effettuare le chiamate successive per la corretta tabella pivot. Per le richieste in seguito ad un'azione da parte di un utente ho utilizzato la seguente funzione \verb|sendAction()| che in modo similare alla precedente esegue una richiesta HTTP POST all'API della tabella di un JSON che contiene le seguenti informazioni:
\begin{itemize}
	\item \verb|axis|: può essere "R" o "C" (righe o colonne);
	\item \verb|path|: array identificativo della cella su cui è stata effettuata l'azione.
\end{itemize}
L'implementazione di \verb|sendAction()| è la seguente: 
\begin{lstlisting}[caption={Funzione sendAction()}, label={lst:bodycells}, language=Kotlin]
suspend fun sendAction(body: String, type: String) {
val res: Any? = window.fetch("$url$INSTANCE_KEY/$type", RequestInit(
method = "POST",
body = body,
credentials = RequestCredentials.Companion.INCLUDE,
headers = {
json("Accept" to "application/json")
json("Content-Type" to "application/json")
}))
.await()
.json()
.await()
}
\end{lstlisting}
\subsection{Sprint Review}
Niente da segnalare.

\subsection{Backlog refinement}
Il Product Backlog non è stato modificato.

\subsection{Problemi riscontrati}
La realizzazione dell'adapter è stata rallentata da alcuni cambiamenti della struttura dell'API da parte dell'azienda. I rallentamenti hanno causato ad una lunghezza dello sprint più elevata. La realizzazione dell'adapter ha infatti anche occupato parte della settimana riservata al quarto sprint.

\newpage

\section{S4: Caricamento parziale}
\subsection{Sprint Review}
Niente da segnalare.

\subsection{Backlog refinement}
Il Product Backlog non è stato modificato.

\subsection{Sprint Review}
\subsection{Backlog refinement}
\subsection{Problemi riscontrati}

