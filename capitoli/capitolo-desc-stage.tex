% !TEX encoding = UTF-8
% !TEX TS-program = pdflatex
% !TEX root = ../tesi.tex

%**************************************************************
\chapter{Codifica}
\label{cap:codifica}
%**************************************************************

\section{Sprint di sviluppo}
La codifica del prodotto, come descritto nel capitolo 2 è stata suddivisa in sprint di sviluppo che corrispondono a circa 4-5 giorni lavorativi dove vengono implementati parte delle user stories indicate nel product backlog. Per ogni sprint verrano descritti lo Sprint Backlog e le Soluzioni che sono state implementate e i seguenti eventi: \textbf{Sprint Review} e \textbf{Backlog refinement}.
Infine verranno elencati i possibili ritardi, i problemi riscontrati e le soluzioni trovate.

\subsection{Realizzazione della struttura dei dati e dello stato Redux}
\subsubsection{Sprint Backlog}
\subsubsection{Soluzioni implementate}
\subsubsection{Sprint Review}
\subsubsection{Backlog refinement}
\subsubsection{Problemi riscontrati}

\subsection{Realizzazione dei componenti grafici e container}
\subsubsection{Sprint Backlog}
\subsubsection{Soluzioni implementate}
\subsubsection{Sprint Review}
\subsubsection{Backlog refinement}
\subsubsection{Problemi riscontrati}

\subsection{Realizzazione del parser JSON e dell'adapter}
\subsubsection{Sprint Backlog}
\subsubsection{Soluzioni implementate}
\subsubsection{Sprint Review}
\subsubsection{Backlog refinement}
\subsubsection{Problemi riscontrati}

\subsection{Realizzazione caricamento parziale}
\subsubsection{Sprint Backlog}
\subsubsection{Soluzioni implementate}
\subsubsection{Sprint Review}
\subsubsection{Backlog refinement}
\subsubsection{Problemi riscontrati}

