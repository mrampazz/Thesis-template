% !TEX encoding = UTF-8
% !TEX TS-program = pdflatex
% !TEX root = ../tesi.tex

%**************************************************************
\chapter{Codifica}
\label{cap:codifica}
%**************************************************************

La codifica del prodotto, come descritto nel capitolo 2 è stata suddivisa in sprint di sviluppo che corrispondono a circa 4-5 giorni lavorativi dove vengono implementati parte delle user stories indicate nel \emph{product backlog}. Per ogni \emph{sprint} verrano identificate lo \emph{sprint backlog}, le soluzioni che sono state implementate e i seguenti eventi: \emph{Sprint Review} e \emph{Backlog refinement}.
Infine verranno elencati i possibili ritardi, i problemi riscontrati e le soluzioni trovate. Ogni sprint è stato identificato da un codice univoco e da un titolo. In questo progetto gli sprint individuati sono i seguenti:
\begin{itemize}
	\item \textbf{S1: Struttura dello stato e dell'architettura Redux};
	\item \textbf{S2: Componenti grafici e container Redux};
	\item \textbf{S3: Parser JSON e adapter};
	\item \textbf{S4: Caricamento parziale}.
\end{itemize}

\section{S1: Struttura dello stato e dell'architettura Redux}
\textbf{Durata:} \textit{5 giorni} \\
Nel primo sprint di sviluppo sono stati implementati i requisiti considerati più importanti per porre delle solide basi dell'applicazione; in particolare la progettazione e codifica della struttura dello stato e l'architettura di Redux.

\subsection{Sprint Backlog}
In particolare sono stati implementati questi requisiti.
\begin{longtable} {
		|>{}p{10mm}| 
		|>{}p{90mm}|
		|>{}p{15mm}|
		|>{}p{15mm}|
		|>{}p{15mm}|
		>{}p{0mm}}
	\hline
	\textbf{Id} & \textbf{Descrizione} & \textbf{Tipo} \\ \hline
	R1.0 & Definizione dello stato dell'applicazione & O \\ \hline
	R1.0.1 & Sviluppo TableState        & O\\ \hline
	R1.0.2 & Sviluppo NodeDimensions    & O\\ \hline
	R1.0.3 & Sviluppo BodyCells         & O\\ \hline
	R1.0.4 & Sviluppo HeaderAction      & O\\ \hline
	R1.0.5 & Sviluppo NodeActionType    & O\\ \hline
	R1.1   & Definizione architettura Redux & O\\ \hline
	R1.1.1 & Sviluppo TableStateSlice    & O\\ \hline
	R1.1.2 & Sviluppo di Thunk & O\\ \hline
	R1.1.3 & Sviluppo delle Actions & O\\ \hline
	R1.1.4 & Sviluppo dei Reducers & O\\ \hline
	\caption{Sprint backlog - 1}
\end{longtable}

\subsection{Soluzioni implementate}
Ho realizzato i seguenti \verb|package|:
\begin{itemize}
	\item \verb|redux|
	\item \verb|redux.slices|
	\item \verb|redux.thunks|
	\item \verb|redux.state|
	\item \verb|entities|
\end{itemize}

\subsubsection{redux.slice}
\begin{lstlisting}[caption={TableStateSlice}, label={lst:tablestateslice}, language=Kotlin]
object TableStateSlice {
	data class State(
	  val cols: ArrayList<ArrayList<DimensionsNode>> = ArrayList(),
	  val rows: ArrayList<ArrayList<DimensionsNode>> = ArrayList(),
	  val cells: ArrayList<ArrayList<BodyCells>> = ArrayList(),
	  val rowActions: ArrayList<ArrayList<HeaderAction>> = ArrayList(),
	  val colActions: ArrayList<ArrayList<HeaderAction>> = ArrayList(),
	  val isLoading: Boolean = false,
	)
	
	private val initTableState = InitState()
	fun initTable() : RThunk = initTableState
	
	class UpdateCells(val cells: ArrayList<ArrayList<BodyCells>>): RAction
	class UpdateRows(val rows: ArrayList<ArrayList<DimensionsNode>>): RAction
	class UpdateCols(val cols: ArrayList<ArrayList<DimensionsNode>>): RAction
	class SetIsLoading(val b: Boolean): RAction
	class UpdateRowActions(val n: ArrayList<ArrayList<HeaderAction>>): RAction
	class UpdateColActions(val n: ArrayList<ArrayList<HeaderAction>>): RAction
	
	fun reducer(state: State = State(), action: RAction) : State {
		return when (action) {
			is UpdateCells -> state.copy(cells = action.cells)
			is UpdateRows -> state.copy(rows = action.rows)
			is UpdateCols -> state.copy(cols = action.cols)
			is SetIsLoading -> state.copy(isLoading = action.b)
			is UpdateRowActions -> state.copy(rowActions = action.n)
			is UpdateColActions -> state.copy(colActions = action.n)
			else -> state
		}
	}
}
\end{lstlisting}

\subsubsection{redux.thunks}
La classe \verb|InitState| si occuperà di riempire lo stato della tabella con il risultato della richiesta HTTP non ancora implementata in questo \emph{sprint}.
\begin{lstlisting}[caption={InitState}, label={lst:initState}, language=Kotlin]
class InitState : RThunk {
	override fun invoke(dispatch: (RAction) -> WrapperAction, getState: () -> AppState): WrapperAction {
		val mainScope = MainScope()
		mainScope.launch {
			// val res : TableState = fetchCreavistaJson()
			dispatch(TableStateSlice.UpdateRows(res.rows))
			dispatch(TableStateSlice.UpdateCols(res.cols))
			dispatch(TableStateSlice.UpdateCells(res.cells))
			dispatch(TableStateSlice.UpdateRowActions(res.rowAction))
			dispatch(TableStateSlice.UpdateColActions(res.colAction))
		}
		return nullAction
	}
}
\end{lstlisting}

\subsubsection{redux.state}
Lo stato dell'applicazione è definito in modo che sia completamente modulare. Infatti la \verb|data class AppState| contiene un'istanza dello stato di \verb|TableStateSlice|. In questo modo in un futuro, se si vorrà espandere questo componente si potrà semplicemente modificare questo file aggiungendo alla funzione \verb|rootReducer()| e a \verb|data class AppState| una referenza del nuovo \emph{slice}.
\begin{lstlisting}[caption={AppState}, label={lst:appstate}, language=Kotlin]
data class AppState (
  val tableState: TableStateSlice.State = TableStateSlice.State()
)

// funzioni per collegare AppState a TableStateSlice
fun rootReducer(): Reducer<AppState, RAction> = getRootReducers(
  mapOf(
    AppState::tableState to TableStateSlice::reducer
  )
)

fun <S, A, R> getRootReducers(reducers: Map<KProperty1<S, R>, Reducer<*, A>>): Reducer<S, A> {
	return combineReducers(reducers.mapKeys { it.key.name })
}
\end{lstlisting}

\subsubsection{entities}
\paragraph*{TableState} \mbox{} \\
L'entità \verb|TableState| è la \verb|data class| che contiene l'intero stato dell'applicazione. Al suo interno sono presenti tutte le informazioni necessarie per la corretta visualizzazione della tabella pivot.
\begin{lstlisting}[caption={TableState}, label={lst:tablestate}, language=Kotlin]
data class TableState(
  val cols: ArrayList<ArrayList<DimensionsNode>>,
  val rows: ArrayList<ArrayList<DimensionsNode>>,
  val cells: ArrayList<ArrayList<BodyCells>>,
  val rowAction: ArrayList<ArrayList<HeaderAction>>,
  val colAction: ArrayList<ArrayList<HeaderAction>>
)
\end{lstlisting}

\paragraph*{DimensionsNode} \mbox{} \\
L'entità \verb|DimensionsNode| è la \verb|data class| che contiene le informazioni riguardanti una cella d'intestazione della tabella pivot.
\begin{lstlisting}[caption={DimensionsNode}, label={lst:dimensionsnode}, language=Kotlin]
data class DimensionsNode(
  var id: String,
  var label: String,
  var level: Int? = null,
  var childDepth: Int? = null,
  var path: List<String>? = null,
  var actionType: NodeActionType = NodeActionType.NULL,
  var isChild: Boolean = false
)
\end{lstlisting}

\paragraph*{HeaderAction} \mbox{} \\
L'entità \verb|HeaderAction| è la \verb|data class| che contiene le informazioni riguardanti i pulsanti che si occupano di aprire intere colonne o righe di dimensioni nella \verb|TableActionUI|.
\begin{lstlisting}[caption={HeaderAction}, label={lst:headeraction}, language=Kotlin]
data class HeaderAction(
  val actionType: NodeActionType,
  val dim: Int,
  val depth: Int,
)
\end{lstlisting}

\paragraph*{NodeActionType} \mbox{} \\
L'entità \verb|HeaderAction| è una \verb|enum class| che definisce il tipo di azione che può essere eseguita da una cella (apertura, chiusura o nessuna azione).
\begin{lstlisting}[caption={NodeActionType}, label={lst:nodeactiontype}, language=Kotlin]
enum class NodeActionType(val type: String) {
  EXPAND("E"),
  COLLAPSE("C"),
  NULL("")
}
\end{lstlisting}

\paragraph*{BodyCells} \mbox{} \\
L'entità \verb|BodyCells| è la \verb|data class| che contiene le informazioni riguardanti le celle che contengono i dati della tabella.
\begin{lstlisting}[caption={BodyCells}, label={lst:bodycells}, language=Kotlin]
data class BodyCells(
  val value: Int,
  val cPath: List<String>,
  val rPath: List<String>
)
\end{lstlisting}

\subsection{Sprint Review}
Implementare correttamente l'architettura Redux è risultato difficoltoso a causa della mia inesperienza con la libreria e la mancanza di documentazione relativa a \verb|kotlin-redux|. In particolare definire correttamente lo \emph{Slice} dello stato ha richiesto molte ore di studio e ricerca dei wrapper di Kotlin per Redux. Questi problemi non hanno però causato rallentamenti nell'implementazione degli altri requisiti dello \emph{Sprint Backlog}.

\subsection{Backlog refinement}
Il Product Backlog non è stato modificato.

\newpage

\section{S2: Componenti grafici e container Redux}
\textbf{Durata:} \textit{5 giorni} \\
Nel secondo \emph{sprint} di sviluppo ho lavorato sull'interfaccia grafica. Quindi in questo \emph{sprint} mi sono occupato della codifica dei componenti React e i relativi container Redux.

\subsection{Sprint Backlog}
\begin{longtable} {
		|>{}p{10mm}| 
		|>{}p{90mm}|
		|>{}p{15mm}|
		|>{}p{15mm}|
		|>{}p{15mm}|
		>{}p{0mm}}
	\hline
\textbf{Id} & \textbf{Descrizione} & \textbf{Tipo} \\ \hline
R2.0   & \textbf{Definizione dei componenti} & O\\ \hline
R2.1   & Sviluppo dei componenti React 		  & O\\ \hline
R2.1.1 & Sviluppo di TableView                & O\\ \hline
R2.1.2 & Sviluppo di TableHeaderView          & O\\ \hline
R2.1.3 & Sviluppo di TableSidebarView         & O \\ \hline
R2.1.4 & Sviluppo di TableBodyView            & O\\ \hline
R2.1.5 & Sviluppo di TableActionUI            & O\\ \hline
R2.1.6 & Sviluppo di TableLabelView            & O\\ \hline
R2.2   & Sviluppo dei container Redux         & O  \\ \hline
R2.2.2 & Sviluppo di TableController          & O   \\ \hline
R2.2.3 & Sviluppo di TableHeaderController    & O      \\ \hline
R2.2.4 & Sviluppo di TableSidebarController   & O      \\ \hline
R2.2.5 & Sviluppo di TableBodyController      & O   \\ \hline
R2.2.6 & Sviluppo di TableLabelController   & O      \\ \hline
R2.2.7 & Sviluppo di TableActionUIController      & O   \\ \hline
\caption{Sprint backlog - 2}
\end{longtable}

\subsection{Soluzioni implementate}
Le componenti grafiche realizzate sono le seguenti: \\
\begin{minipage}{\linewidth}
\includegraphics[scale=0.60]{./immagini/table-pivot.png}
\captionof{figure}{Screenshot del componente realizzato}
\end{minipage}
\\
\\
Per rendere l'interfaccia utente chiara e utilizzabile da un dispositivo mobile ho implementato la funzionalità di nascondere le \emph{dimensioni d'analisi della tabella} per ampliare lo spazio relativo ai dati in caso ce ne fosse bisogno. Inoltre ogni cella d'intestazione può essere cliccata per attivare l'azione, questo perchè cliccare un pulsante molto piccolo da un dispositivo mobile è risultato difficile nei test manuali effettuati.

\subsubsection{redux.container}
Per ogni componente che richiedeva un collegamento con lo stato di Redux ho realizzato un \emph{wrapper} che permette di mappare lo stato di Redux alle \emph{props} di un componente React. I \emph{wrapper} sono stati definiti nel seguente modo:
\begin{lstlisting}[caption={ReduxContainer}, label={lst:reduxcontainer}, language=Kotlin]
// campi dato dello stato da mappare alle props
private interface StateProps : RProps {
	var rows: ArrayList<DimensionsNode>
	...
}

// funzioni per modificare lo stato da mappare alle props
private interface DispatchProps: RProps {
	var updateRows: (rows: ArrayList<DimensionsNode>) -> Unit
	...
}

val reactComponent: RClass<RComponentProps> =
rConnect<AppState, RAction, WrapperAction, RProps, StateProps, DispatchProps, RComponentProps>(
mapStateToProps = { state, _ ->
	rows = state.tableState.rows
},
mapDispatchToProps = { dispatch, _ ->
	updateRows = { rows -> dispatch(TableStateSlice.UpdateRows(rows)) }
}
)(TableActionUI::class.js.unsafeCast<RClass<RComponentProps>>())
\end{lstlisting}

\subsection{Sprint Review}
All'inizio dello sprint ho realizzato i componenti grafici con una scrittura funzionale, tuttavia dopo aver implementato i container Redux ho notato la presenza di alcuni limitazioni nell'uso di Redux e componenti funzionale di React. In particolare le funzioni dei wrapper \emph{Redux} di Kotlin non permettono di eseguire la connessione tra i componenti funzionali React e lo stato di Redux. Per risolvere questo problema ho quindi riscritto i componenti grafici con una scrittura a classi. Questo problema non ha causato ritardi gravi. 

\subsection{Backlog refinement}
Il Product Backlog non è stato modificato.

\newpage

\section{S3: Parser JSON e adapter}
\textbf{Durata:} \textit{8 giorni} \\
Nel terzo \emph{sprint} ho realizzato le funzioni per eseguire richieste HTTP e le ho integrate con le funzioni eseguite nel primo \emph{sprint}. In seguito ho realizzato le \verb|data class @Serializable| per eseguire il \emph{parsing} del json. Infine ho scritto le funzioni che mi hanno permesso di tradurre i dati ricevuti dall'\emph{API} nelle strutture dati definite nel primo \emph{sprint}.
 
\subsection{Sprint Backlog}
\begin{longtable} {
		|>{}p{10mm}| 
		|>{}p{90mm}|
		|>{}p{15mm}|
		|>{}p{15mm}|
		|>{}p{15mm}|
		>{}p{0mm}}
	\hline
	\textbf{Id} & \textbf{Descrizione} & \textbf{Tipo} \\ \hline
	R3.0   & \textbf{Sviluppo parser per JSON dell'API}         & O\\ \hline
	R3.1   & Sviluppo data class @Serializable 	   & O\\ \hline
	R3.1.1 & Sviluppo Gruppo4JSON & O\\ \hline
	R3.1.2 & Sviluppo Gruppo4Data & O\\ \hline
	R3.1.3 & Sviluppo Gruppo4Filter & O\\ \hline
	R3.1.4 & Sviluppo Gruppo4Actions & O\\ \hline
	R3.1.5 & Sviluppo Gruppo4Node & O\\ \hline
	R3.2   & \textbf{Sviluppo adapter da JSON a TableState} & O\\ \hline
	R3.2.1   & Sviluppo funzione convertListOfGruppo4Node() & O\\ \hline
	R3.2.2   & Sviluppo funzione convertTree() & O \\ \hline
	R3.2.3   & Sviluppo funzione convertListOfGruppo4Actions() & O\\ \hline
	R3.2.4   & Sviluppo funzione convertCells() & O\\ \hline
	
	R4.0 & \textbf{Sviluppo funzioni per effettuare richieste HTTP}  & O    \\ \hline
	R4.1   & Sviluppo funzione fetch() & O     \\ \hline
	R4.2   & Sviluppo funzione sendAction() & O    \\ \hline
	\caption{Sprint backlog - 3}
\end{longtable}

\subsection{Soluzioni implementate}
\subsubsection{Gestione delle richieste all'API}
Il primo passo per utilizzare dei dati reali all'interno della tabella pivot è stato quello di realizzare una funzione per effettuare richieste HTTP alla \emph{API} di Gruppo4. Per farlo ho utilizzato l'implementazione in Kotlin di \verb|window.fetch|. La funzione risultante è la seguente:
\begin{lstlisting}[caption={Funzione fetch()}, label={lst:fetch}, language=Kotlin]
suspend fun fetch() {
  val res = window.fetch(url, RequestInit(
    method = "GET",
    credentials = RequestCredentials.Companion.INCLUDE,
    headers = {
      json("Accept" to "application/json")
      json("Content-Type" to "application/json")
    }))
    .await()
    .json()
    .await()

   // Codice identificativo dell'istanza della tabella pivot necessario
   // per le chiamate successive
   INSTANCE_KEY = (res as kotlin.js.Json)["InstanceKey"] as String?
}
\end{lstlisting}

\noindent
La seguente funzione effettua una richiesta HTTP GET alla \emph{API} di Gruppo4 la quale ritorna il json risultante e una chiave necessaria per effettuare le chiamate successive per la corretta tabella pivot. Quando un utente esegue un'azione all'interno della tabella questa deve essere comunicata all'API. Per ottenere questa funzionalità ho utilizzato la seguente funzione \verb|sendAction()| che in modo similare alla precedente esegue una richiesta HTTP POST all'\emph{API} della tabella di un json che contiene le seguenti informazioni:
\begin{itemize}
	\item \verb|axis|: può essere "R" o "C" (righe o colonne);
	\item \verb|path|: array identificativo della cella su cui è stata effettuata l'azione.
\end{itemize}
L'implementazione di \verb|sendAction()| è la seguente: 
\begin{lstlisting}[caption={Funzione sendAction()}, label={lst:sendaction}, language=Kotlin]
suspend fun sendAction(body: String, type: String) {
    val res: Any? = window.fetch("$url$INSTANCE_KEY/$type", RequestInit(
      method = "POST",
      body = body,
      credentials = RequestCredentials.Companion.INCLUDE,
      headers = {
        json("Accept" to "application/json")
        json("Content-Type" to "application/json")
      }))
     .await()
     .json()
     .await()
}
\end{lstlisting}

\subsubsection{Entità @Serializable}
Le entità \verb|@Serializable| sono state scritte per ogni oggetto json all'interno del json in arrivo dall'\emph{API}. Per non dilungare questa sezione con codice molto simile mostrerò solo la struttura di una entità:
\begin{lstlisting}[caption={data class Gruppo4Action}, label={lst:gruppo4action}, language=Kotlin]
@Serializable
data class Gruppo4Action(
  @SerialName("Action")
  val action: String? = null,
  @SerialName("Dim")
  val dim: Int? = null,
  @SerialName("Depth")
  val depth: Int? = null,
  @SerialName("Path")
  val path: Array<String>? = null
)
\end{lstlisting}
\subsubsection{Funzioni adapter}
Di seguito indico le funzioni principali per eseguire la conversione tra le entità \verb|@Serializable| e le entità dello stato. Dalle seguenti funzioni si può vedere l'utilizzo delle \emph{higher order functions} per semplificare il più possibile il codice.

\begin{lstlisting}[caption={Funzione convertListOfGruppo4Node()}, label={lst:gruppo4node}, language=Kotlin]
fun convertListOfGruppo4Node(list: ArrayList<Gruppo4Node>?): ArrayList<DimensionsNode>? {
	return list?.map { it.toDimensionsNode() }?.toCollection(ArrayList())
}
\end{lstlisting}

\begin{lstlisting}[caption={Funzione convertListOfGruppo4Actions()}, label={lst:convertlistofgruppo4actions}, language=Kotlin]
fun convertListOfGruppo4Actions(list: ArrayList<Gruppo4Action>, limit: Int): ArrayList<ArrayList<HeaderAction>> {
	val res: ArrayList<ArrayList<HeaderAction>> = ArrayList()
	var index = 0
	while (index < limit) {
		res.add(index, ArrayList())
		index++
	}
	
	list.groupBy { it.dim }.entries.map {
		val tmpList = it.value.map { c -> c.toHeaderAction() }.toCollection(ArrayList())
		val limit = tmpList.last().depth
		res.set(it.key!!, normaliseListAction(tmpList, limit))
	}
	
	return res
}
\end{lstlisting}

\begin{lstlisting}[caption={Funzione convertCells()}, label={lst:convertcells}, language=Kotlin]
fun convertCells(list: Array<Array<Double>>, rowsPaths: Array<Array<String>>? , colsPaths: Array<Array<String>>?): ArrayList<ArrayList<BodyCells>>? {
	val tmp: ArrayList<ArrayList<BodyCells>> = ArrayList()
	var cellTmp: ArrayList<BodyCells> = ArrayList()
	if (rowsPaths == null || colsPaths == null) {
		return null
	}
	for ((indx, items) in list.withIndex()) {
		val rPath = rowsPaths[indx]
		for ((index, item) in items.withIndex()) {
			cellTmp.add(
			BodyCells(
			value = item.toInt(),
			cPath = colsPaths[index],
			rPath = rPath
			)
			)
		}
		tmp.add(cellTmp)
		cellTmp = ArrayList()
	}
	return tmp
}
\end{lstlisting}

\begin{lstlisting}[caption={Funzione convertTree()}, label={lst:converttree}, language=Kotlin]
fun convertTree(node: DimensionsNode?, level: Int, currentPath: List<String>, depth: Int): List<DimensionsNode?> =
if (node == null) {
	listOf()
} else {
	val subDimNode = node.subDim
	val action = createActionFor(node.children)
	val newNode = node.setLevel(level).setPath(currentPath).setActionType(action).setDepth(depth)
	(
	listOf(newNode) +
	(
	subDimNode?.flatMap {
		getRow(
		it.setLevel(level + 1),
		level + 1,
		currentPath + it.id,
		if (!node.isChild) depth else 0
		)
	} ?: listOf()
	) +
	(
	node.children?.flatMap {
		getRow(
		it.setLevel(level).setIsChild(true),
		level,
		currentPath.dropLast(1) + it.id,
		depth + 1
		)
	} ?: listOf()
	)
	).sortedWith(compareBy { it?.level })
}
\end{lstlisting}

\begin{lstlisting}[caption={Funzione parseJSON()}, label={lst:parsejson}, language=Kotlin]
fun parseJSON(res: String): TableState {
  val obj = Json.decodeFromString<Gruppo4Json>(res)

  val tableData: ArrayList<ArrayList<BodyCells>> = convertCells(obj.cells, obj.rows?.paths, obj.cols?.paths)!!
  val rowTree = convertListOfGruppo4Node(obj.rows?.tree)
  val colTree = convertListOfGruppo4Node(obj.cols?.tree)

  val rowCells = getCellsFromTree(rowTree !!)
  val colCells = getCellsFromTree(colTree !!)

  val rowNormalisedCells = normaliseList(rowCells)
  val colNormalisedCells = normaliseList(colCells)

  val rowAction = gruppo4actionsToActionType(obj.rows?.actions !!, rowCells.groupBy { it?.level }.size)
  val colAction = gruppo4actionsToActionType(obj.cols?.actions !!, colCells.groupBy { it?.level }.size)

  val rows = rowCells.toCollection(ArrayList())
  val cols = colCells.toCollection(ArrayList())

  return TableState(
    cells = tableData,
    cols = colNormalisedCells,
    rows = rowNormalisedCells,
    colTree = cols,
    rowTree = rows,
    rowAction = rowAction,
    colAction = colAction
  )
}
\end{lstlisting}

\subsection{Sprint Review}
La realizzazione dell'adapter è stata rallentata da cambiamenti della struttura dell'API da parte dell'azienda. In particolare il cambiamento di alcuni oggetti JSON che hanno implicato un'ampliamento delle entità \verb|@Serializable|. Inoltre la struttura ad albero che contiene le dimensioni d'analisi della tabella è stata modifica estensivamente. La nuova struttura identifica per ogni nodo due tipologie di figli: i nodi graficamente sotto di esso e i nodi graficamente laterali ad esso. Questi cambiamenti hanno implicato una riscrittura delle funzioni dell'adapter per ovviare a questo cambiamento. \\
La realizzazione dell'adapter ha quindi occupato parte della settimana riservata al quarto sprint.

\subsection{Backlog refinement}
Il Product Backlog non è stato modificato.

\newpage

\section{S4: Caricamento parziale}
\textbf{Durata:} \textit{4 giorni} \\
Nell'ultimo sprint finale sono state realizzate le funzioni che riguardano il caricamento parziale dei dati della tabella. Tuttavia per alcuni problemi riguardanti l'API fornita da Gruppo4 è stato impossibile realizzare in modo completo la funzionalità del caricamento parziale.

\subsection{Sprint Backlog}
\begin{longtable} {
		|>{}p{10mm}| 
		|>{}p{90mm}|
		|>{}p{15mm}|
		|>{}p{15mm}|
		|>{}p{15mm}|
		>{}p{0mm}}
	\hline
	\textbf{Id} & \textbf{Descrizione} & \textbf{Tipo} \\ \hline
	R5.0 & \textbf{Sviluppo caricamento parziale}  & O  \\ \hline
	R5.1   & Sviluppo componente infinteScroller &   O   \\ \hline
	R5.2   & Sviluppo funzioni per aggiornare struttura ad albero & O     \\ \hline
\caption{Sprint backlog - 4}
\end{longtable}

\subsection{Soluzioni implementate}
Il seguente componente utilizza funzioni della programmazione funzionale fornite da React come: \verb|useState| e \verb|useEffect| per controllare quando l'utente ha raggiunto la fine della tabella in modo da effettuare una successiva chiamata all'API.

\begin{lstlisting}[caption={Componente infiniteScroller}, label={lst:infinitescroller}, language=Kotlin]
val infiniteScroller = functionalComponent<InfiniteScrollerProps> { props ->
	val (isFetchingVertical, setIsFetchingVertical) = useState(false)
	val (isFetchingHorizontal, setIsFetchingHorizontal) = useState(false)
	
	val isScrolling = { e: Event ->
		val el = document.getElementById("pivot-table-body");
		if (el != null && el.scrollTop.toInt() + el.clientHeight == el.scrollHeight) {
			setIsFetchingVertical(true)
		}
		
		if (el != null && el.scrollLeft.toInt() + el.clientWidth == el.scrollWidth) {
			setIsFetchingHorizontal(true)
		}
	}
	
	useEffect(listOf(isFetchingVertical)) {
		if (isFetchingVertical) {
			// eseguire chiamata api
		}
	}
	
	useEffect(listOf(isFetchingHorizontal)) {
		if (isFetchingHorizontal) {
			// eseguire chiamata api
		}
	}
	div {
		attrs.id = "pivot-table-body"
		attrs.onScrollFunction = isScrolling
		props.children()
	}
}
\end{lstlisting}

Per quanto riguarda il caricamento parziale all'apertura o chiusura di un nodo di una dimensione ho realizzato le seguenti funzioni:
\begin{lstlisting}[caption={Funzioni update struttura albero}, label={lst:updateTree}, language=Kotlin]
fun updateNode(tree: ArrayList<DimensionsNode>, node: DimensionsNode): ArrayList<DimensionsNode> {
	if (tree.size > 1) {
		tree.map {
			findNode(it, node)
		}
	} else {
		findNode(tree[0], node)
	}
	return tree
}

fun findNodeChildren(current: DimensionsNode, node: DimensionsNode) {
	current.children?.map { child ->
		findNode(child, node)
	}
}

fun findNode(current: DimensionsNode, node: DimensionsNode) {
	if (current.subDim.isNullOrEmpty())
	return
	
	if (current.path == node.path) {
		current.replace(node)
		return
	}
	
	current.subDim!!.map {
		findNode(it, node)
		it.children?.map { child ->
			if (child.subDim.isNullOrEmpty())
			  findNodeChildren(child, node)
			else
			  findNode(child, node)
		}
	}
}
\end{lstlisting}
\subsection{Sprint Review}
Nell'ultimo sprint sono stati riscontrati dei problemi per quanto riguarda la realizzazione del caricamento parziale durante lo scroll di un utente. Il componente che si occuperà di realizzare la chiamata all'\emph{API} è stato implementato, tuttavia l'\emph{API} di Gruppo4 non presenta ancora le informazioni necessarie per implementare questa funzionalità; l'azienda mi ha informato che queste modifiche potranno essere implementate solo dopo la fine del tirocinio curriculare. Le funzione e il componente richiesto dall'azienda è stato comunque implementato.

\subsection{Backlog refinement}
Il Product Backlog non è stato modificato.
