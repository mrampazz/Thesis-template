% !TEX encoding = UTF-8
% !TEX TS-program = pdflatex
% !TEX root = ../tesi.tex

%**************************************************************
\chapter{Codifica}
\label{cap:codifica}
%**************************************************************

\section{Sprint di sviluppo}
La codifica del prodotto, come descritto nel capitolo 2 è stata suddivisa in sprint di sviluppo che corrispondono a circa 4-5 giorni lavorativi dove vengono implementati parte delle user stories indicate nel product backlog. Per ogni sprint verrano descritti lo Sprint Backlog e le Soluzioni che sono state implementate e i seguenti eventi: \textbf{Sprint Review} e \textbf{Backlog refinement}.
Infine verranno elencati i possibili ritardi, i problemi riscontrati e le soluzioni trovate. Ogni sprint è stato identificato da un codice univoco e da un titolo. In questo progetto gli sprint individuati sono stati:
\begin{itemize}
	\item \textbf{S1: Struttura dello stato e dell'architettura Redux};
	\item \textbf{S2: Componenti grafici e container Redux};
	\item \textbf{S3: Parser JSON e adapter};
	\item \textbf{S4: Caricamento parziale}.
\end{itemize}

\subsection{S1: Struttura dello stato e dell'architettura Redux}
\textbf{Durata:} \textit{5 giorni} \\
Nel primo sprint di sviluppo sono stati implementati i requisiti considerati più importanti per porre delle solide basi dell'applicazione web; in particolare la progettazione e codifica della struttura dello stato e l'architettura di Redux.
% maybe insert grafico

\subsubsection{Sprint Backlog}
In particolare sono stati implementati questi requisiti.
\begin{longtable} {
		|>{}p{10mm}| 
		|>{}p{90mm}|
		|>{}p{15mm}|
		|>{}p{15mm}|
		|>{}p{15mm}|
		>{}p{0mm}}
	\hline
	\textbf{Id} & \textbf{Descrizione} & \textbf{Tipo} \\ \hline
	R1.0 & Definizione dello stato dell'applicazione & O \\ \hline
	R1.0.1 & Sviluppo TableState        & O\\ \hline
	R1.0.2 & Sviluppo NodeDimensions    & O\\ \hline
	R1.0.3 & Sviluppo BodyCells         & O\\ \hline
	R1.0.4 & Sviluppo HeaderAction      & O\\ \hline
	R1.0.5 & Sviluppo NodeActionType    & O\\ \hline
	R1.1   & Definizione architettura Redux & O\\ \hline
	R1.1.1 & Sviluppo TableStateSlice    & O\\ \hline
	R1.1.2 & Sviluppo di Thunk & O\\ \hline
	R1.1.3 & Sviluppo delle Actions & O\\ \hline
	R1.1.4 & Sviluppo dei Reducers & O\\ \hline
\end{longtable}

\subsubsection{Struttura del progetto}
\dirtree{%
	.1 kotlin.
		.2 redux.
			.3 slices.
				.4 TableStateSlice.kt.
			.3 thunks.
				.4 RThunk.kt.
				.4 InitState.kt.
			.3 state.
				.4 Index.kt.
		.2 entities.
			.3 StateEntities.kt.
}
\subsubsection{Sprint Review}
Niente da segnalare.

\subsubsection{Backlog refinement}
Il Product Backlog non è stato modificato.

\subsubsection{Problemi riscontrati}
Implementare correttamente l'architettura Redux è risultato difficoltoso a causa della mia inesperienza con la libreria e la mancanza di documentazione relativa a \verb|kotlin-redux|. Questi problemi non hanno però causato rallentamenti nell'implementazione degli altri requisiti dello Sprint Backlog.

\newpage

\subsection{S2: Componenti grafici e container Redux}
\textbf{Durata:} \textit{4 giorni} \\
Nel secondo sprint di sviluppo ho lavorato sull'interfaccia grafica e quindi i componenti React e i relativi container Redux.

\subsubsection{Sprint Backlog}
\begin{longtable} {
		|>{}p{10mm}| 
		|>{}p{90mm}|
		|>{}p{15mm}|
		|>{}p{15mm}|
		|>{}p{15mm}|
		>{}p{0mm}}
	\hline
R2.0   & \textbf{Definizione e pianificazione dei componenti} & O\\ \hline
R2.1   & Sviluppo dei componenti React 		  & O\\ \hline
R2.1.1 & Sviluppo di TableView                & O\\ \hline
R2.1.2 & Sviluppo di TableHeaderView          & O\\ \hline
R2.1.3 & Sviluppo di TableSidebarView         & O \\ \hline
R2.1.4 & Sviluppo di TableBodyView            & O\\ \hline
R2.2   & Sviluppo dei container Redux         & O  \\ \hline
R2.2.2 & Sviluppo di TableController          & O   \\ \hline
R2.2.3 & Sviluppo di TableHeaderController    & O      \\ \hline
R2.2.4 & Sviluppo di TableSidebarController   & O      \\ \hline
R2.2.5 & Sviluppo di TableBodyController      & O   \\ \hline
\end{longtable}
\subsubsection{Struttura del progetto}
\dirtree{%
	.1 kotlin.
	.2 redux.
	.3 slices.
	.4 TableStateSlice.kt.
	.3 thunks.
	.4 RThunk.kt.
	.4 InitState.kt.
	.3 state.
	.4 Index.kt.
	.3 container.
	.4 TableViewController.kt.
	.4 TableHeaderViewController.kt.
	.4 TableSidebarViewController.kt.
	.4 TableBodyViewController.kt.
	.4 TableActionUIController.kt.
	.4 TableLabelViewController.kt.
	.2 entities.
	.3 StateEntities.kt.
	.2 view.
	.3 TableView.kt.
	.3 TableHeaderView.kt.
	.3 TableSidebarView.kt.
	.3 TableBodyView.kt.
	.3 TableActionUI.kt.
	.3 TableLabelView.kt.
}

\subsubsection{Sprint Review}
Niente da segnalare.

\subsubsection{Backlog refinement}
Il Product Backlog non è stato modificato.

\subsubsection{Problemi riscontrati}
Niente da segnalare.
\newpage

\subsection{S3: Parser JSON e adapter}
\subsubsection{Sprint Backlog}
\subsubsection{Soluzioni implementate}
\subsubsection{Sprint Review}
Niente da segnalare.

\subsubsection{Backlog refinement}
Il Product Backlog non è stato modificato.

\subsubsection{Problemi riscontrati}
La realizzazione dell'adapter è stata rallentata da alcuni cambiamenti della struttura dell'API da parte dell'azienda. I rallentamenti hanno causato ad una lunghezza dello sprint più elevata. La realizzazione dell'adapter ha infatti anche occupato parte della settimana riservata al quarto sprint.

\newpage

\subsection{S4: Caricamento parziale}
\subsubsection{Sprint Review}
Niente da segnalare.

\subsubsection{Backlog refinement}
Il Product Backlog non è stato modificato.

\subsubsection{Sprint Review}
\subsubsection{Backlog refinement}
\subsubsection{Problemi riscontrati}

