
%**************************************************************
% Acronimi
%**************************************************************
\renewcommand{\acronymname}{Acronimi e abbreviazioni}

\newacronym[description={\glslink{apig}{Application Program Interface}}]
    {api}{API}{Application Program Interface}

\newacronym[description={\glslink{umlg}{Unified Modeling Language}}]
    {uml}{UML}{Unified Modeling Language}

%**************************************************************
% Glossario
%**************************************************************
%\renewcommand{\glossaryname}{Glossario}

\newglossaryentry{apig}
{
    name=\glslink{api}{API},
    text=Application Program Interface,
    sort=api,
    description={in informatica con il termine \emph{Application Programming Interface API} (ing. interfaccia di programmazione di un'applicazione) si indica ogni insieme di procedure disponibili al programmatore, di solito raggruppate a formare un set di strumenti specifici per l'espletamento di un determinato compito all'interno di un certo programma. La finalità è ottenere un'astrazione, di solito tra l'hardware e il programmatore o tra software a basso e quello ad alto livello semplificando così il lavoro di programmazione}
}


\newglossaryentry{dataflowg}
{
    name=Dataflow,
    text=dataflow,
    sort=dataflow,
    description={Il dataflow è un paradigma di programmazione basato sull'idea di disconnettere gli attori computazionali in "pipelines" che possono eseguire azioni in modo concorrente. Nell'ambito di questo progetto usando "dataflow" si intende il modo in cui i dati vengono trasmessi tra gli attori computazionali}
}

\newglossaryentry{webappg}
{
	name=Web App,
	text=Web App,
	sort=webapp,
	description={Una applicazione web (o web app) è un software che viene eseguito su un web server. Rispetto ad applicazioni che vengono eseguite localmente sul sistema operativo, le applicazioni web possono essere eseguite da un utente mediante un browser e una connessione internet}
}

\newglossaryentry{scrumg}
{
	name=SCRUM,
	text=SCRUM,
	sort=scrum,
	description={Scrum è un framework agile per lo sviluppo e la distribuzione e la manutenzione di prodotti complessi. Esso è stato realizzato per team di sviluppo ristretti (meno di 10 membri) che dividino il loro lavoro in obiettivi che possono essere completati all'interno di iterazioni con un tempo limitato che si chiamano "sprint". Alla fine di uno "sprint" il team di sviluppo partecipa alla riunione di "sprint review" per dimostrare il lavoro fatto, infine uno "sprint retrospective" per ottenere un miglioramento continuo}
}

\newglossaryentry{hofg}
{
	name=Funzioni di ordine superiore / Higher order functions,
	text=Funzioni di ordine superiore / Higher order functions,
	sort=hofg,
	description={
		In matematica ed in informatica, una funzione di ordine superiore (o higher order function) è una funzione che presenta almeno una di queste caratteristiche:
		\begin{itemize}
			\item accetta come argomento una o più funzioni (parametri procedurali);
			\item ritorna una funzione come risultato.
		\end{itemize}
		Un esempio nella matematica è l'operatore di derivazione dato che viene mappata una funzione ad un'altra funzione}
}

\newglossaryentry{sideeffectsg}
{
	name=Side effects,
	text=Side effects,
	sort=sideeffects,
	description={In informatica una funziona, operazione o espressione viene detta di avere "side effect" se modifica variabili di stato al di fuori del suo ambiente locale. In particolare se c'è un effetto osservabile al di fuori del valore che viene ritornato ("main effect") all'invocatore della funzione, operazione o espressione}
}

\newglossaryentry{vcsg}
{
	name=VCS,
	text=VCS,
	sort=vcs,
	description={VCS o "Version Control System". In ingegneria del software, il controllo del versionamento è una classe di sistemi responsabili della gestione di cambiamenti nel codice sorgente di un software. Ogni cambiamento (revisione) viene identificato attraverso un numero o un codice, esso viene associato ad un'orario e alla persona che ha eseguito la modifica. Ogni revisione può essere confrontata, restaurata e, con alcuni tipi di file, combinata (merged)}
}

\newglossaryentry{itsg}
{
	name=ITS,
	text=ITS,
	sort=its,
	description={ITS o "Issue Tracking System" è un pacchetto software che si occupa di gestire e mantenere una lista di attività (issues) da completare. Un ITS viene utilizzato principalmente durante la collaborazione in team di sviluppo grandi, tuttavia può essere utilizzato da individui per gestire il proprio tempo e/o la loro produttività}
}

\newglossaryentry{ideg}
{
	name=IDE,
	text=IDE,
	sort=ide,
	description={IDE o "Integrated Development Environment" è un software per sviluppare applicazioni che combina strumenti per lo sviluppatore in un'interfaccia grafica comune}
}

\newglossaryentry{buildautomationg}
{
	name=Build automation,
	text=Build automation,
	sort=buildautomation,
	description={Con Build automation si intende il processo di automatizzare la creazione di un software e i processi associati ad esso tra i quali: compilazione del codice sorgente, esecuzione di test automatici e il "packaging" del codice binario risultante}
}

\newglossaryentry{purefunctionsg}
{
	name=Funzioni pure,
	text=Funzioni pure,
	sort=purefunctions,
	description={Una funzione è definita come "pura" quando il valore di output è determinato solamente dall'input della funzione senza nessun "side effect" osservabile}
}

\newglossaryentry{propsg}
{
	name=Props,
	text=Props,
	sort=props,
	description={I componenti di React sono delle funzioni Javascript, esse accettano dei valori di input chiamati "props". Quindi le "props" sono i dati che possono essere passati come argomento ad un componente React}
}



